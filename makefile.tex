% !TEX encoding = IsoLatin9

%%%%%%%%%%%%%%%%%%%%% SECTION 1
\section{Makefile}
\begin{frame}
  \begin{columns}
    \column{4.8cm}
    \tableofcontents[currentsection,hideothersubsections]
    \column{7cm}
    
  \end{columns}
\end{frame}


\begin{frame}[fragile]
\frametitle{L'outil \Verb|make|}
\begin{block}{}
L'outil \Verb|make| est un programme pr�sent sous Linux permettant
de d�clencher un certains nombre de commandes \Verb|shell| � partir d'un
fichier nomm� \Verb|Makefile| ou \Verb|makefile|.
\end{block}
\Verb|make| permet dans votre cas d'automatiser la compilation en 
exploitant les avantages de la compilation s�par�e.

Proc�dez comme suit : 
\begin{enumerate}
\item D�finir les r�gles de compilation dans le fichier \Verb|Makefile|
\item Au moment de la compilation, invocation en entrant dans le terminal : \Verb|make|
\item La compilation se fait, en recompilant uniquement les fichiers n�cessaires (ceux qui ont �t�
modifi�s depuis la derni�re compilation, que l'on rep�re grace � la date de modification des fichiers).
\end{enumerate}

\end{frame}

\begin{frame}[fragile]
\frametitle{Structure du \Verb|Makefile|}
Un makefile est un ensemble de r�gles d�finies par :
\begin{codeblock}{}
\vspace{-.3cm}
\lstset{escapeinside={��}}
\lstset{basicstyle=\scriptsize}
\lstset{moredelim=**[is][\color{red}]{@}{@}}
\begin{codeC}
nom_cible : liste_dependances
<TAB>commandes
\end{codeC}
\vspace{-.3cm}
\end{codeblock}

\begin{itemize}
\item \Verb|nom_cible| : nom du fichier � g�n�rer
\item \Verb|liste_dependances| : liste des fichiers
permettant la g�n�ration de la cible
\item \Verb|commandes| (\red{Obligatoirement pr�c�d� d'une tabulation}) :
Commande de compilation permettant de g�n�rer la cible.
\end{itemize}

\end{frame}

\begin{frame}[fragile]
\frametitle{Exemple}
\begin{codeblock}{Makefile}
\vspace{-.3cm}
\lstset{escapeinside={��}}
\lstset{basicstyle=\scriptsize}
\lstset{moredelim=**[is][\color{red}]{@}{@}}
\begin{codeC}
matrices : matrices.o matricesio.o operations.o 
<tab>gcc matrices.o matricesio.o operations.o -o matrices

matrices.o : matrices.c operations.h matricesio.h commun.h 
<tab>gcc -c matrices.c

matricesio.o : matricesio.c matricesio.h commun.h 
<tab>gcc -c matricesio.c

operations.o : operations.c operations.h commun.h 
<tab>gcc -c operations.c

clean :
<tab>rm -f *.o matrices
\end{codeC}
\vspace{-.3cm}
\end{codeblock}

\begin{termblock}{}
%\vspace{-.3cm}
\lstset{escapeinside={��}}
\lstset{basicstyle=\scriptsize}
\lstset{moredelim=**[is][\color{red}]{@}{@}}
\begin{lstlisting}
>> make clean
>> make matrices
\end{lstlisting}
\vspace{-.3cm}
\end{termblock}

\end{frame}