% !TEX encoding = IsoLatin9

%%%%%%%%%%%%%%%%%%%%% SECTION 1
\section{Les fonctions standards}
\begin{frame}
  \begin{columns}
    \column{4.8cm}
    \tableofcontents[currentsection,hideothersubsections]
    \column{7cm}
    
  \end{columns}
  
\end{frame}

\begin{frame}[fragile]
\frametitle{La fonction \bvrb|rand|}
\begin{block}{Prototype}
\bvrb|int rand();|
\end{block}
\begin{itemize}
\setlength\itemsep{1em}
\item Renvoie un nombre al�atoire entre \Verb|0|
et \Verb|RAND_MAX|.
\item Utilisation :
\begin{columns}
\column{0.5\textwidth}
\begin{codeblock}{}
\vspace{-.3cm}
\lstset{escapeinside={��}}
\lstset{basicstyle=\scriptsize}
\begin{codeC}
#include <stdio.h>
#include <stdlib.h>

int main() {
  int alea ;
  alea = rand() ;
  printf("%d",alea);
}
\end{codeC}
\vspace{-.3cm}
\end{codeblock}
\column{0.4\textwidth}
Valeur type de \Verb|RAND_MAX| :
2~147~483~647
\end{columns}
\end{itemize}
\begin{alertblock}{}
Dans cet exemple, la valeur renvoy�e par \bvrb|rand|
sera la m�me � chaque ex�cution du programme
\end{alertblock}

\end{frame}

\begin{frame}[fragile]
\frametitle{Comment changer de nombre al�atoire � chaque ex�cution ?}
\begin{block}{La solution}
Il faut initialiser de mani�re diff�rente le g�n�rateur al�toire
� chaque ex�cution.\\
Astuce : Utiliser l'heure d'ex�cution.
\end{block}
\begin{codeblock}{}
\vspace{-.3cm}
\lstset{escapeinside={��}}
\lstset{basicstyle=\scriptsize}
\begin{codeC}
#include <stdio.h>
#include <stdlib.h>
#include <time.h>

int main() {
  int alea ;
  srand(time(NULL));
  alea = rand() ;
  printf("%d",alea);
}
\end{codeC}
\vspace{-.3cm}
\end{codeblock}
\end{frame}

\begin{frame}[fragile]
\frametitle{Les fonctions math�matiques \bvrb|math.h|}
\begin{codeblock}{}
\vspace{-.3cm}
\lstset{escapeinside={��}}
\lstset{basicstyle=\scriptsize}
\begin{codeC}
#include <math.h>
\end{codeC}
\vspace{-.3cm}
\end{codeblock}

\begin{figure}
\centering
\begin{tabular}{|c|c|c|c|c|c|}
\hline
\multicolumn{3}{|c|}{Donctions trigonom�triques} &
\multicolumn{3}{|c|}{Fonctions �l�mentaires} \\
\hline
\bvrb|sin(x)| & \bvrb|cos(x)| & \bvrb|tan(x)| &
\bvrb|exp(x)| & \bvrb|pow(x,y)| & \bvrb|sqrt(x)| \\
\hline
\bvrb|asin(x)| & \bvrb|acos(x)| & \bvrb|atan(x)| &
\bvrb|log(x)| & \bvrb|ceil(x,y)| & \bvrb|floor(x)| \\
\hline
\bvrb|sinh(x)| & \bvrb|cosh(x)| & \bvrb|tanh(x)| &
\bvrb|log10(x)| & \bvrb|fabs(x,y)| & \bvrb|fmod(x)| \\
\hline
\end{tabular}
\end{figure}

\begin{alertblock}{N�cessite l'option \bvrb|-lm| � la compilation}
\Verb|gcc -o prg prg.c -lm|
\end{alertblock}

\end{frame}