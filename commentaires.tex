% !TEX encoding = IsoLatin9

%%%%%%%%%%%%%%%%%%%%% SECTION 1
\section{Les commentaires }
\begin{frame}
  \begin{columns}
    \column{4.8cm}
    \tableofcontents[currentsection,hideothersubsections]
    \column{7cm}
    \centering{
      \includegraphics[width=6cm]{fig/code.jpg}
      
      \textit{Eagleson's Law: "Any code of your own that you haven't looked at for six or more months, might as well have been written by someone else."}\\
      \small{
        \hfill Anonyme
      }
    }
  \end{columns}
  
\end{frame}

\begin{frame}[fragile]
\frametitle{Les commentaires}
\begin{itemize}
\item Pour mettre le reste d'une ligne en commentaire, on le pr�c�de de \bvrb|//| :
\begin{codeblock}{}
\lstset{escapeinside={��}}
\begin{codeC}
int a ; // a est un entier
// fin de la d�claraion des variables
\end{codeC}
\end{codeblock}
\item Pour mettre tout un bloc en commentaire, on l'encadre entre \bvrb|/*| et \bvrb|*/| :
\begin{codeblock}{}
\lstset{escapeinside={��}}
\begin{codeC}
#include <stdio.h>

/* Ce qui suit est la fonction principale de hello.c
Elle affiche : hello world*/

int main()
{...
\end{codeC}
\end{codeblock}
\end{itemize}
\end{frame}

\begin{frame}
\frametitle{Remarques sur les commentaires}
\begin{block}{Pourquoi utiliser les commentaires ?}
\begin{itemize}
\item Maintenir le code (le modifier apr�s une semaine, un mois,
un an, ...)\\
\item Travailler en �quipe\\
\item Expliquer une partie d'un algorithme\\
\item Peut permettre le d�boggage (voir cours n$^o$6)\\
\item G�n�rer une documentation automatique (doxygen, javadoc, ...)
\end{itemize}
\end{block}
\begin{alertblock}{Attention}
Commenter un programme fait partie du travail du programmeur (c'est-�-dire : \textbf{vous}).
\end{alertblock}

\end{frame}