% !TEX encoding = IsoLatin9

%%%%%%%%%%%%%%%%%%%%% SECTION 1
\section{Conversion automatique et "cast"}
\begin{frame}
  \begin{columns}
    \column{4.8cm}
    \tableofcontents[currentsection,hideothersubsections]
    \column{7cm}
      \centering{
      \includegraphics[width=7cm]{fig/conversion.jpg}

      \hfill \textit{La conversion de Saint Paul} (vers 1690)\\
      \hfill Luca Giordano (1634-1705)\\
      \hfill Mus�e des Beaux-Arts de Nancy
    }
  \end{columns}
  
\end{frame}

\begin{frame}[fragile]
\frametitle{Conversions implicites}
\begin{block}{}
Si une expression contient des op�randes de types diff�rents (expressions mixte),
des conversions se font automatiquement.
\end{block}
Hi�rarchie :\\
\bvrb|int| $\leftarrow$ \bvrb|long| $\leftarrow$ \bvrb|float| $\leftarrow$ \bvrb|double|

\begin{codeblock}{}
\lstset{escapeinside={��}}
%\lstset{basicstyle=\scriptsize}
\begin{codeC}
int a ; float x;
a*x ; // sera de type float
a*a + a*x // sera de type float
\end{codeC}
\end{codeblock}
\begin{alertblock}{}
Les conversions se font au fur et � mesure de l'�valuation, op�ration par op�ration
\end{alertblock}
\end{frame}

\begin{frame}[fragile]
\frametitle{Conversions explicites (cast)}
\begin{block}{}
Il est possible de forcer une conversion d'une expression quelconque
dans un type de son choix par la syntaxe :\\
\centering{
\bvrb|(�textit�type�) �textit�operande� ;|\\
\bvrb|(�textit�type�) �textit�expression� ;|\\
}
\end{block}
\begin{codeblock}{}
\lstset{escapeinside={��}}
%\lstset{basicstyle=\scriptsize}
\begin{codeC}
float x = 5.2 ;
int a = 5 ;
(int) x ; // = 5
(float) a ; // = 5.0
(int) (x+a) ; // =10
\end{codeC}
\end{codeblock}

\end{frame}

\begin{frame}[fragile]
\frametitle{Remarques}
\begin{itemize}
\item Une division entre entiers est par d�faut une division enti�re (euclidienne).
\begin{codeblock}{}
\lstset{escapeinside={��}}
%\lstset{basicstyle=\scriptsize}
\begin{codeC}
float x ;
x = 3/2 ; //x = 1
x = ((float) a) / ((float) b) ; //x=1.5
\end{codeC}
\end{codeblock}
\pause
\item La conversion se fait avant l'affectation
\begin{codeblock}{}
\lstset{escapeinside={��}}
%\lstset{basicstyle=\scriptsize}
\begin{codeC}
int a = 3 , b = 2 ; float x ;
x = a / b ; // x = 1.0
x = (float) (a/b) ; // x = 1.0
x=  ((float a) / ((float) b) ; //x=1.5
\end{codeC}
\end{codeblock}
\end{itemize}

\end{frame}