% !TEX encoding = IsoLatin9

%%%%%%%%%%%%%%%%%%%%% SECTION 1
\section{Passage de param�tres au \texttt{main}}
\begin{frame}
  \begin{columns}
    \column{4.8cm}
    \tableofcontents[currentsection,hideothersubsections]
    \column{7cm}
    
  \end{columns}
\end{frame}
  
\begin{frame}[fragile]
\frametitle{La fonction \bvrb|main|}
\begin{itemize}
\setlength\itemsep{1em}
\item Un code C doit contenir obligatoirement une fonction \bvrb|main|.
\item Le \bvrb|main| est le point d'entr�e de l'ex�cutable.
\end{itemize}
\begin{block}{}
Il est possible de communiquer des informations � l'ex�cutable.
\end{block}

Exemple : le programme \Verb|gedit| est un ex�cutable, on peut lui passer
en argument le nom du fichier � ouvrir.

\begin{termblock}{Ouvre l'�diteur de texte avec un fichier vide :}
%\vspace{-.3cm}
\lstset{escapeinside={��}}
\lstset{basicstyle=\scriptsize}
\begin{lstlisting}
>> gedit
\end{lstlisting}
\vspace{-.3cm}
\end{termblock}

\begin{termblock}{Ouvre le fichier \Verb|hello.c| :}
%\vspace{-.3cm}
\lstset{escapeinside={��}}
\lstset{basicstyle=\scriptsize}
\begin{lstlisting}
>> gedit hello.c
\end{lstlisting}
\vspace{-.3cm}
\end{termblock}

\end{frame}

\begin{frame}[fragile]
\frametitle{Arguments du \bvrb|main|}
\begin{block}{}
Il est possible de r�cup�rer les arguments pass�s
� l'ex�cutable gr�ce aux arguments du main.
\end{block}
\begin{codeblock}{}
\vspace{-.3cm}
\lstset{escapeinside={��}}
%\lstset{basicstyle=\scriptsize}
\begin{codeC}
int main (int argc, char *argv[]){
\end{codeC}
\vspace{-.3cm}
\end{codeblock}

\begin{itemize}
\setlength\itemsep{1em}
\item \bvrb|argc :| contient le nombre d'arguments
pass�s � l'ex�cutable (nombre de mots dans la ligne
de commande).
\item \bvrb|argv :| tableau des arguments.\\
\red{Le premier
argument est le nom de l'ex�cutable.}
\end{itemize}
\end{frame}

\begin{frame}[fragile]
\frametitle{Comment �a marche ?}
\begin{itemize}
\item Dans le terminal :
\begin{termblock}{}
\vspace{-.3cm}
\lstset{escapeinside={��}}
\lstset{basicstyle=\scriptsize}
\begin{lstlisting}
>> monprog arg1 arg2
\end{lstlisting}
\vspace{-.3cm}
\end{termblock}

\item Dans le code :
\begin{codeblock}{}
\vspace{-.3cm}
\lstset{escapeinside={��}}
%\lstset{basicstyle=\scriptsize}
\begin{codeC}
int main (int argc, char *argv[]){
\end{codeC}
\vspace{-.3cm}
\end{codeblock}

\item En m�moire :
\begin{figure}
\begin{tikzpicture}[
auto,
node distance = 1em,
block/.style  = {rectangle, inner sep = 0pt},
]
\node (argc) [block] {\Verb|argc|};
\node (argc3) [block,right = of argc] {
  \begin{tabular}{|c|}
    \hline
    3 \\
    \hline
    \end{tabular}
    };
\node (argv) [block,below = of argc] {\Verb|argv|};
\node (argcp) [block,right = of argv] {
  \begin{tabular}{|c|}
    \hline
    \phantom{3} \\
    \hline
    \end{tabular}
    };
\node(argc0) [block,right= of argcp] {
 \begin{tabular}{|c|}
    \hline
    \Verb|argv[0]| \\
    \hline
    \end{tabular}
    };
\node(argc0v) [block,right = of argc0] {
\begin{tabular}{|c|c|c|c|c|c|c|c|}
    \hline
    m & o & n & p & r & o & g & \Verb|\0|\\
    \hline
    \end{tabular}
    };

\node(argc1) [block,below= of argc0,yshift = 0.8 em] {
 \begin{tabular}{|c|}
    \hline
    \Verb|argv[1]| \\
    \hline
    \end{tabular}
    };
\node(argc1v) [block,right = of argc1] {
\begin{tabular}{|c|c|c|c|c|}
    \hline
    a & r & g & 1 & \Verb|\0|\\
    \hline
    \end{tabular}
    };

\node(argc2) [block,below= of argc1,yshift = 0.8em] {
 \begin{tabular}{|c|}
    \hline
    \Verb|argv[2]| \\
    \hline
    \end{tabular}
    };
\node(argc2v) [block,right = of argc2] {
\begin{tabular}{|c|c|c|c|c|}
    \hline
    a & r & g & 2 & \Verb|\0|\\
    \hline
    \end{tabular}
    };

\draw [black,->] (argc0.east) -- (argc0v.west) ;
\draw [black,->] (argc1.east) -- (argc1v.west) ;
\draw [black,->] (argc2.east) -- (argc2v.west) ;
\draw [black,->] (argcp.east) -- (argc0.west) ;

\end{tikzpicture}
\end{figure}

\end{itemize}
\end{frame}

\begin{frame}[fragile]
\frametitle{Un exemple}

\begin{codeblock}{\Verb|monprog.c|}
\vspace{-.3cm}
\lstset{escapeinside={��}}
\lstset{basicstyle=\scriptsize}
\begin{codeC}
int main (int argc, char * argv[]) {
 int i;
 for (i=1;i<argc;i++)
 {
  printf("Argument %d : %s\n",i,argv[i]);
 }
}
\end{codeC}
\vspace{-.3cm}
\end{codeblock}

\begin{termblock}{Test d'ex�cution}
%\vspace{-.3cm}
\lstset{escapeinside={��}}
\lstset{basicstyle=\scriptsize}
\begin{lstlisting}
>> monprog fic1.txt 100
Argument 1 : fic1.txt
Argument 2 : 100
\end{lstlisting}
\vspace{-.3cm}
\end{termblock}
\end{frame}

\begin{frame}
\frametitle{Utilisation}
\begin{itemize}
\setlength\itemsep{1.5em}
\item Permet de transmettre des informations
du shell au programme.
\item Utile dans les scripts shell qui appellent
plusieurs ex�cutables.
\item Permet de passer certains param�tres
� vos ex�cutable (nom de fichiers, identifiants, etc.)
\end{itemize}
\end{frame}