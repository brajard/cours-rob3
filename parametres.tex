% !TEX encoding = IsoLatin9

%%%%%%%%%%%%%%%%%%%%% SECTION 1
\section{Passage et renvoi de param�tres}
\begin{frame}
  \begin{columns}
    \column{4.8cm}
    \tableofcontents[currentsection,hideothersubsections]
    \column{7cm}
    
  \end{columns}
  
\end{frame}

\begin{frame}[fragile]
\frametitle{Entr�es et sorties}
\begin{block}{Prototype d'une fonction}
\bvrb|�textit�type_retour nom_fonction�(�textit�arguments�);|
\end{block}
\vspace{1cm}
\begin{itemize}
\item Les entr�es de la fonction sont appel�s \red{arguments}.
\item \bvrb|�textit�type_retour�| est le type renvoy� par la fonction.
Une fonction (en C) ne peut renvoyer qu'une valeur de ce type.
\end{itemize}
\end{frame}

\begin{frame}[fragile]
\frametitle{Les sorties}
\framesubtitle{L'instruction \bvrb|return|}
\begin{block}{}
Une fonction retourne une valeur � l'appellent par 
l'instruction \bvrb|return|.
\end{block}
\begin{itemize}
\item Syntaxe : \bvrb|return(�textit�expression�);|
\item Exemples :
\begin{columns}
\column{.3\textwidth}
\begin{codeblock}{}
\vspace{-.3cm}
\lstset{escapeinside={��}}
\lstset{basicstyle=\scriptsize}
\begin{codeC}
return (-1) ;
\end{codeC}
\vspace{-.3cm}
\end{codeblock}

\column{.3\textwidth}
\begin{codeblock}{}
\vspace{-.3cm}
\lstset{escapeinside={��}}
\lstset{basicstyle=\scriptsize}
\begin{codeC}
return (2*z + 3) ;
\end{codeC}
\vspace{-.3cm}
\end{codeblock}

\column{.3\textwidth}
\begin{codeblock}{}
\vspace{-.3cm}
\lstset{escapeinside={��}}
\lstset{basicstyle=\scriptsize}
\begin{codeC}
return (2 * f(1+z));
\end{codeC}
\vspace{-.3cm}
\end{codeblock}
\end{columns}
\item La valeur de retour est convertie selon le type de la
valeur de retour pr�cis� dans le prototype de la fonction.
\item Parfois, le compilateur �met un avertissement (\Verb|Warning|)
si ce n'est pas le cas.
\item La fonction appelante n'utilise pas forc�ment la valeur de retour.
\end{itemize}
\end{frame}