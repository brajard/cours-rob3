% !TEX encoding = IsoLatin9
\section{Tableaux statiques}
\begin{frame}
  \begin{columns}
    \column{4.8cm}
    \tableofcontents[currentsection,hideothersubsections]
    \column{7cm}
    \centering{
      \includegraphics[width=4cm]{fig/kelly.jpg}
      
      \textit{``Should array indices start at 0 or 1 ? 
My compromise of 0.5 was rejected without, I thought, proper consideration''}\\
      \small{
        \hfill Stan Kelly-Bootle (1929-2014)\\
               \hfill informaticien, auteur-compositeur}
    }
  \end{columns}
  
\end{frame}

\begin{frame}
\frametitle{G�n�ralit�s}
\begin{itemize}
  \setlength\itemsep{1em}
\item Un tableau est un ensemble d'�l�ments de m�me type
d�sign�s par un identificateur unique.\\
\item Chaque �l�ment est rep�r� par un \red{indice} pr�cisant
sa position au sein de l'ensemble.\\
\item Il se d�clare en m�me temps que les autres variables.\\
\item Il existe deux grandes familles de tableaux : \\
\begin{itemize}
\item Les tableaux unidimensionnels (vecteurs)\\
\item Les tableaux multidimensionnels (matrices)\\
\end{itemize}
\end{itemize}
\end{frame}

\begin{frame}[fragile]
\frametitle{Tableaux unidimensionnels}
\begin{itemize}
\item On doit r�server un emplacement m�moire pour un certain
nombre d'�l�ments.\\
{\centering{
\bvrb|�textit�type nom_tableau�[ �textit�nb_elements� ];|\\
}}
\begin{itemize}
\item \bvrb|�textit�type�| est le type des �lements du tableau \bvrb|�textit�nom_tableau�|\\
\item \bvrb|�textit�nb_elements�| est le nombre d'�l�ments que contient le tableau.\\
\end{itemize}

\item Conventionnellement, \red{la premi�re position porte le num�ro 0}.
Les indices vont donc de \verb|0| � \Verb|nb_elements-1|
\end{itemize}
\begin{codeblock}{}
%\vspace{-.3cm}
\lstset{escapeinside={��}}
%\lstset{basicstyle=\scriptsize}
\begin{codeC}
int tab[10] ; // tab est un tableau de 10 entiers
float zz[5] ; // zz est un tableau de 5 r�els
char y[12] ; // y est un tableau de 12 caract�res
\end{codeC}
\end{codeblock}
\end{frame}