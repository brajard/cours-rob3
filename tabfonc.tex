% !TEX encoding = IsoLatin9

%%%%%%%%%%%%%%%%%%%%% SECTION 1
\section{Passage d'un tableau en argument}
\begin{frame}
  \begin{columns}
    \column{4.8cm}
    \tableofcontents[currentsection,hideothersubsections]
    \column{7cm}
    
  \end{columns}
  
\end{frame}

\begin{frame}[fragile]
\frametitle{Passage d'un tableau en argument}
\begin{itemize}
\setlength\itemsep{1em}
\item Pour un tableau � une dimension, on passe
le tableau et sa taille\\
\bvrb|�textit�type_retour nom_fonction�(float �textit�Tab�[N], int n);|

\item Pour un tableau � deux dimensions :\\
\bvrb|�textit�type_ret nom_fonct�(float �textit�Tab�[N1][N2], int n1, int n2);|

\end{itemize}

\begin{alertblock}{}
\Verb|N|, \Verb|N1| et \Verb|N2| ne sont \red{pas}
des variables, ce sont des expressions constantes ou des
identificateurs pour le pr�compileur via un 
\bvrb|#define|.
\end{alertblock}
\end{frame}


\begin{frame}[fragile]
\frametitle{Exemple}
\begin{columns}
\column{0.65\textwidth}

\begin{codeblock}{}
\vspace{-.3cm}
\lstset{escapeinside={��}}
\lstset{basicstyle=\scriptsize}
\begin{codeC}
#define N1 3
#define N2 2

void affiche(int T[N1][N2], int n1, int n2);

int main() {
  int Tab[N1][N2]={1,2,3,4,5,6};
  affiche(Tab,N1,N2);
}

void affiche(int T[N1][N2], int n1, int n2)
{
  int i,j;
  for (i=0 ; i < n1 ; i++) {
    for (j=0 ; j < n2 ; j++) {
      printf("%d\t",Tab[i][j]);
    }
    printf("\n");
  }
}
\end{codeC}
\vspace{-.3cm}
\end{codeblock}

\column{0.3\textwidth}
Qu'affiche le programme ?
\end{columns}

\end{frame}

\begin{frame}[fragile]
\frametitle{Tol�rance}
\begin{block}{}
L'indication de la taille de la premi�re dimension est facultatif.
\end{block}

\begin{itemize}
\setlength\itemsep{1em}
\item Tableau � une dimension :\\
\bvrb|�textit�type_retour nom_fonction�(float �textit�Tab�[], int n);|
\begin{codeblock}{}
\vspace{-.3cm}
\lstset{escapeinside={��}}
\lstset{basicstyle=\scriptsize}
\begin{codeC}
void affiche(int T[], int n);
\end{codeC}
\vspace{-.3cm}
\end{codeblock}

\item Pour un tableau � deux dimensions :\\
\bvrb|�textit�type_ret nom_fonct�(float �textit�Tab�[][N2], int n1, int n2);|
\begin{codeblock}{}
\vspace{-.3cm}
\lstset{escapeinside={��}}
\lstset{basicstyle=\scriptsize}
\begin{codeC}
void affiche(int T[][2], int n1, int n2);
\end{codeC}
\vspace{-.3cm}
\end{codeblock}

\end{itemize}

\end{frame}

\begin{frame}
\frametitle{Tableau entr�e/sortie}
\begin{block}{}
Un tableau est un cas particulier d'argument
qui peut �tre modifi�.

C'est un passage par r�f�rence (voir cours sur les pointeurs).
\end{block}

\begin{alertblock}{}
Si le tableau est modifi� dans la fonction appel�, il sera
aussi modifi� pour la fonction appelante.
\end{alertblock}

\end{frame}

\begin{frame}[fragile]
\frametitle{Exemple}
\begin{columns}
\column{0.65\textwidth}

\begin{codeblock}{}
\vspace{-.3cm}
\lstset{escapeinside={��}}
\lstset{basicstyle=\scriptsize}
\begin{codeC}
#define N 3

void affiche(int Tab[], int n);
void init0 (int Tab[],int n); 

int main() {
  int T[N]={1,2,3};
  affiche(T,N);
  init0(T,N);
  affiche(T,N);
}

void init0 (int Tab[],int n)
{
  int i;
  for (i=0 ; i < n ; i++) {
   Tab[i]=0;
  }
}
\end{codeC}
\vspace{-.3cm}
\end{codeblock}

\column{0.3\textwidth}
Qu'affiche le programme ?
\end{columns}

\end{frame}

