% !TEX encoding = IsoLatin9

%%%%%%%%%%%%%%%%%%%%% SECTION 1
\section{Adresse et pointeurs}
\begin{frame}
  \begin{columns}
    \column{4.8cm}
    \tableofcontents[currentsection,hideothersubsections]
    \column{7cm}
      \centering{
      \includegraphics[width=5cm]{fig/postman.jpg}

      \hfill \textit{Portrait of a Postman} (vers 1900-1912 ??)\\
      \hfill Thomas Patterson \\
      \hfill The British Postal Museum \& Archive
    }
  \end{columns}
  
\end{frame}

\begin{frame}[fragile]
\frametitle{Variable et adresse m�moire}
\begin{block}{}
La m�moire d'un ordinateur est organis�e en suite de cases rep�r�es par 
une adresse.
\begin{itemize}
\item Chaque case m�moire a une adresse et un contenu.
\end{itemize}
\end{block}

Les variables d'un programme sont stock�es en m�moire et poss�dent
une valeur.


\begin{columns}
\column{0.2\textwidth}

\begin{codeblock}{}
\vspace{-.3cm}
\lstset{escapeinside={��}}
\lstset{basicstyle=\scriptsize}
\begin{codeC}
int n=3;
\end{codeC}
\vspace{-.3cm}
\end{codeblock}
\end{columns}
\vspace{1em}
\begin{columns}
\column{0.6\textwidth}

\begin{tabular}{|l|c|c|c|c|c|}
\hline
Adresse & ... & 4584 & 4585 & 4586 & ...\\
\hline
Valeur & ... &     & 3 & & \\
\hline
\multicolumn{3}{c}{} & \multicolumn{1}{c}{\Verb|n|}& \multicolumn{2}{c}{}\\
\end{tabular}

\column{0.35\textwidth}
\Verb|n| est � l'adresse 4585

\end{columns}
\end{frame}

\begin{frame}[fragile]
\frametitle{L'op�rateur \textcolor{bluegreen}{\&}}

\begin{block}{}
L'op�rateur \bvrb|&| permet de retrouver l'adresse d'une variable.
\end{block}

\begin{columns}
\column{0.2\textwidth}

\begin{codeblock}{}
\vspace{-.3cm}
\lstset{escapeinside={��}}
\lstset{basicstyle=\scriptsize}
\begin{codeC}
int n=3;
\end{codeC}
\vspace{-.3cm}
\end{codeblock}
\end{columns}

\vspace{1em}
\centering
\begin{tabular}{|l|c|c|c|c|c|}
\hline
Adresse & ... & 4584 & 4585 & 4586 & ...\\
\hline
Valeur & ... &     & 3 & & \\
\hline
\multicolumn{3}{c}{} & \multicolumn{1}{c}{\Verb|n|}& \multicolumn{2}{c}{}\\
\end{tabular}
\vspace{1em}

\begin{tabular}{ll}
\Verb|n| & contient 3 \\
\Verb|&n| & contient 4585 \\
\end{tabular}

\end{frame}

\begin{frame}[fragile]
\frametitle{Les pointeurs}
\begin{block}{}
Un pointeur est une variable contenant l'adresse d'une case m�moire.
\end{block}
\begin{itemize}
\item D�claration :\\
\vspace{0.5em}
\begin{columns}
\column{0.2\textwidth}
\begin{codeblock}{}
\vspace{-.3cm}
\lstset{escapeinside={��}}
\lstset{basicstyle=\scriptsize}
\begin{codeC}
int *pn;
\end{codeC}
\vspace{-.3cm}
\end{codeblock}

\column{0.75\textwidth}
\begin{tabular}{|l|c|c|c|c|c|c|}
\hline
Adresse & ... & 4584 & 4585 & 4586 & ...&6208\\
\hline
Valeur & ... &     & 3 & & &\\
\hline
\multicolumn{3}{c}{} & \multicolumn{1}{c}{\Verb|n|}& \multicolumn{2}{c}{} &\multicolumn{1}{c}{\Verb|pn|}\\
\end{tabular}

\end{columns}

\item Affection :\\
\vspace{0.5em}
\begin{columns}
\column{0.2\textwidth}
\begin{codeblock}{}
\vspace{-.3cm}
\lstset{escapeinside={��}}
\lstset{basicstyle=\scriptsize}
\begin{codeC}
pn = &n ;
\end{codeC}
\vspace{-.3cm}
\end{codeblock}

\column{0.75\textwidth}
\begin{tabular}{|l|c|c|c|c|c|c|}
\hline
Adresse & ... & 4584 & 4585 & 4586 & ...&6208\\
\hline
Valeur & ... &     & 3 & &  & 4585\\
\hline
\multicolumn{3}{c}{} & \multicolumn{1}{c}{\Verb|n|}& \multicolumn{2}{c}{} &\multicolumn{1}{c}{\Verb|pn|}\\
\end{tabular}

\end{columns}

\end{itemize}
\end{frame}

\begin{frame}[fragile]
\frametitle{L'op�rateur \bvrb|*|}
\begin{block}{}
\begin{itemize}
\item L'op�rateur \bvrb|&| permet de retrouver l'adresse d'une variable.\\
\item L'op�rateur \bvrb|*| permet de retrouver le contenu d'une adresse m�moire. (Op�ration de d�r�f�rencement)\\
\end{itemize}
\end{block}
\centering
\begin{tabular}{|l|c|c|c|c|c|c|}
\hline
Adresse & ... & 4584 & 4585 & 4586 & ...&6208\\
\hline
Valeur & ... &     & 3 & &  & 4585\\
\hline
\multicolumn{3}{c}{} & \multicolumn{1}{c}{\Verb|n|}& \multicolumn{2}{c}{} &\multicolumn{1}{c}{\Verb|pn|}\\
\end{tabular}
\begin{columns}
\column{0.27\textwidth}
\begin{tabular}{ll}
\Verb|n| & contient 3 \\
\Verb|pn| & contient 4585
\end{tabular}

\column{0.27\textwidth}
\begin{tabular}{ll}
\Verb|&n| & contient 4585 \\
\Verb|*pn| & contient 3
\end{tabular}

\column{0.35\textwidth}
\begin{exampleblock}{}
Les �galit�s suivantes sont vraies :\\
\Verb|n == *pn|\\
\Verb|pn == &n | 
\end{exampleblock}

\end{columns}

\end{frame}