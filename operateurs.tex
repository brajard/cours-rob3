% !TEX encoding = IsoLatin9

%%%%%%%%%%%%%%%%%%%%% SECTION 1
\section{Op�rateurs}
\begin{frame}
  \begin{columns}
    \column{4.8cm}
    \tableofcontents[currentsection,hideothersubsections]
    \column{7cm}
    
  \end{columns}
  
\end{frame}

\begin{frame}[fragile]
\frametitle{Rappel : l'op�rateur d'affectaction}
\begin{block}{}
L'op�rateur d'affectation est le signe \bvrb|=|
\end{block}
\begin{columns}
\column{0.45\textwidth}
\begin{codeblock}{}
\lstset{escapeinside={��}}
\begin{codeC}
a = 1 ;  �\tikz[remember picture,baseline = -.5ex]\coordinate(a1);�
x = 2 * 3 + 5 ; �\tikz[remember picture,baseline = -.5ex]\coordinate(x);�
a = a + x ; �\tikz[remember picture,baseline = -.5ex]\coordinate(a2);�
\end{codeC}
\end{codeblock}
\column{0.45\textwidth}
\tikz[remember picture,baseline = -.5ex]\coordinate(la1);
\verb|a| re�oit la valeur 1  ;

\tikz[remember picture,baseline = -.5ex]\coordinate(lx);
\verb|x| re�oit la valeur 11 ;

\tikz[remember picture,baseline = -.5ex]\coordinate(la2);
\verb|a| re�oit la valeur 12  ;

\end{columns}
\begin{tikzpicture}[remember picture,overlay]
\draw (la1) edge[->, gray,thick] (a1);
\draw (lx) edge[->, gray,thick] (x);
\draw (la2) edge[->, gray,thick] (a2);
\end{tikzpicture}

\end{frame}

\begin{frame}[fragile]
\frametitle{Les op�rateurs arithm�tiques}
\begin{block}{}
Les expressions arithm�tiques sont �valu�es de 
la gauche vers la droite en tenant de 
la pr�s�ance (priorit� des op�rateurs)
\end{block}
\begin {columns}
\column{.4\textwidth}
\begin{table}
\centering
\begin{tabular}{|r|p{3cm}|}
\hline
\bvrb|*| & multiplication \\
\hline
\bvrb|/| & division  \\
\hline
\bvrb|+| & addition \\
\hline
\bvrb|-| & soustraction \\
\hline
\bvrb|-| & reste de la division enti�re (modulo) \\
\hline
\end{tabular}
\end{table}
\column{0.55\textwidth}
\begin{codeblock}{}
\lstset{escapeinside={��}}
\begin{codeC}
a+b*c //  a + (b*c)
-c%d //  (-c)%d
a*b + c%d // (a*b)+(c%d)
-a/-b+c //  ((-a)/(-b))+c
a/b/c //  (a/b)/c
\end{codeC}
\end{codeblock}

\end{columns}
\end{frame}

\begin{frame}[fragile]
\frametitle{Les op�rateurs de comparaison}
\begin{block}{}
Ils permettent de comparer entre elles des expressions ou des variables
\end{block}
\begin{columns}
\column{.55\textwidth}
\begin{table}
\centering
\begin{tabular}{|c|p{4cm}|}
\hline
\bvrb|>| &  strictement sup�rieur �\\
\hline
\bvrb|>=| &   sup�rieur ou �gal �\\
\hline
\bvrb|<| &  strictement inf�rieur �\\
\hline
\bvrb|<=| &  inf�rieur ou �gal �\\
\hline
\bvrb|==| &  �gal �\\
\hline
\bvrb|!=| &  diff�rent de\\
\hline
\end{tabular}
\end{table}

\column{.43\textwidth}
\begin{codeblock}{}
\lstset{escapeinside={��}}
\begin{codeC}
x = 5 ;
y = 4 ;
x > y ; //=1 (vrai)
x == y ; //=0 (faux)
\end{codeC}
\end{codeblock}
\end{columns}

\begin{block}{}
Le r�sultat de la comparaison est un entier de valeur:
\begin{itemize}
\item \bvrb|non nulle| si la comparaison est \red{vraie}.\\
\item \bvrb|0| si la comparaison est \red{fausse}.\\
\end{itemize}
\end{block}

\end{frame}

\begin{frame}[fragile]
\frametitle{Les op�rateurs logiques}
\begin{block}{}
Ils effectuent de op�rations logiques (et, ou, ...) entre
expressions ou variables.
\end{block}
\begin{columns}
\column{0.4\textwidth}
\begin{table}
\centering
\begin{tabular}{|c|p{3cm}|}
\hline
\bvrb|&&| & \textbf{et} logique \\
\hline
\bvrb!||! & \textbf{ou} logique \\
\hline
\bvrb|!| & \textbf{non} logique \\
\hline
\end{tabular}
\end{table}
\column{.55\textwidth}
\begin{table}
\centering
\begin{tabular}{|c|c|c|}
\hline
\verb|x| & 0 (faux) & $\neq$ 0 (vrai) \\
\hline
\verb|!x| & $\neq$ 0 (vrai) & 0 (faux) \\
\hline
\end{tabular}
\caption{Table de v�rit� du non logique}

\end{table}
\end{columns}


\end{frame}