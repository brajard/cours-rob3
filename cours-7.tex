\ifdefined\ishandout
\documentclass[handout]{beamer}
\else
\documentclass{beamer}
\fi

\usepackage[frenchb]{babel}
\usepackage[T1]{fontenc}
\usepackage[latin1]{inputenc}
\usepackage{hyperref}
\usepackage{multirow}
\usepackage{listings}
\usepackage{fancyvrb}
\usepackage{tikz}
\usepackage{framed}
\usepackage{algorithm}
\usepackage{algorithmic}
\usepackage{xcolor}
\usepackage{color, colortbl}
\usepackage{handoutWithNotes}
\usepackage{amsmath}
\usetikzlibrary{shapes.geometric}
\usetikzlibrary{positioning}
\usetikzlibrary{shapes.arrows, chains}
\usetikzlibrary{arrows,calc}
\usetikzlibrary{shapes.multipart}
\usepackage{array}
\usetheme{Boadilla}
\definecolor{BlueGreen}{cmyk}{0.85,0,0.33,0}
\definecolor{Gray}{rgb}{0.8,0.8,0.8}

\ifdefined\ishandout
\pgfpagesuselayout{3 on 1 with notes}[a4paper,border shrink=5mm]
\usecolortheme{dove}
\else
\usecolortheme{dolphin}
\fi


\lstnewenvironment{codeC}
{ \lstset{language=C,
    otherkeywords={printf,scanf}}
}
{}

\ifdefined\ishandout
\definecolor{mygreen}{rgb}{0,0,0}
\definecolor{mymauve}{rgb}{0,0,0}
\definecolor{myblue}{rgb}{0,0,0}
\else
\definecolor{mygreen}{rgb}{0,0.6,0}
\definecolor{mymauve}{rgb}{0.58,0,0.82}
\definecolor{myblue}{rgb}{0,0,1}

\fi

\definecolor{mygray}{rgb}{0.5,0.5,0.5}


\lstset{language=C,
% breakatwhitespace=false,         % sets if automatic breaks should only happen at whitespace
%  breaklines=true,                 % sets automatic line breaking
%  captionpos=b,                
commentstyle=\itshape\color{mymauve},
keywordstyle=\bfseries\color{myblue},
%numbers=left,                    % where to put the line-numbers; possible values are (none, left, right)
%  numbersep=8pt,                   % how far the line-numbers are from the code
%  numberstyle=\tiny\color{mygray}, % the style that is used for the line-numbers
  rulecolor=\color{black},         % if not set, the frame-color may be changed on line-breaks within not-black text (e.g. comments (green here))
%  showspaces=false,                % show spaces everywhere adding particular underscores; it overrides 'showstringspaces'
  showstringspaces=false,          % underline spaces within strings only
%  showtabs=false,                  % show tabs within strings adding particular underscores
%  stepnumber=2,                    % the step between two line-numbers. If it's 1, each line will be numbered
  stringstyle=\color{mygreen},     % string literal style
%  tabsize=2 
}
\ifdefined\ishandout
\newcommand{\red}{\textbf}
\else
\newcommand{\red}{\textcolor{red}}
\fi
%\newcommand \emph
%Default size : 12.8 cm * 9.6 cm

\newcommand{\tmark}[1]{\tikz[remember picture, baseline=-.5ex]{\coordinate(#1);}}

\ifdefined\ishandout
\newenvironment<>{codeblock}[1]{%begin
  \setbeamercolor{block title}{fg=black,bg=lightgray!80}%
  \begin{block}{#1}}
  % \begin{codeC}}
  %  {\end{codeC}
{  
\end{block}}

\newenvironment<>{termblock}[1]{
    \setbeamercolor{block title}{fg=black,bg=lightgray!90}%
    \begin{block}{#1}
}
%     \begin{Verbatim}}
{%\end{Verbatim}
\end{block}
}

\definecolor{bluegreen}{RGB}{0,0,0}
%\definecolor{bluegreen}{rgb}{0,0.6,0.8}
\else

\newenvironment<>{codeblock}[1]{%begin
  \setbeamercolor{block title}{fg=darkgray,bg=yellow}%
  \begin{block}{#1}}
  % \begin{codeC}}
  %  {\end{codeC}
{  
\end{block}}

\newenvironment<>{termblock}[1]{
    \setbeamercolor{block title}{fg=white,bg=lightgray}%
    \begin{block}{#1}}
%     \begin{Verbatim}}
{%\end{Verbatim}
\end{block}
}

\definecolor{bluegreen}{RGB}{0,149,182}
%\definecolor{bluegreen}{rgb}{0,0.6,0.8}
\fi

%\newcommand{\output}[1]{
\setbeamertemplate{navigation symbols}{}
\newcommand{\bvrb}{\Verb[commandchars=���,formatcom=\color{bluegreen}]}
\newcommand{\footvrb}{\footnotesize\Verb}
\newcommand{\vrbalert}[2][]{\visible<#1>{#2}}
%%% Commande pour les listes/arbres
\newcommand{\mvide}{\nodepart{one} \nodepart{two}}
\newcommand{\tvide}{\nodepart{one} \nodepart{two} \nodepart{three}}

%%Fin des commandes pour les listes/arbres.


\newcommand<>{\case}[2]{%
\filldraw (#1,#2) rectangle (1+#1,1+#2);}
%%% Param�tres du cours (� r�gler)
%Num�ro du cours
\newcommand{\nb}{7}
\title[cours n�7]{Compilation}
\author[]{julien.brajard@upmc.fr}
\institute[Polytech'UPMC]{Polytech'UPMC}
\date{6 Novembre 2017}
\begin{document}

\begin{frame}
\titlepage
\centering{
\url{https://moodle-sciences.upmc.fr} (cours Informatique G�n�rale MAIN-ROB)}
\end{frame}

\begin{frame}
\frametitle{Plan du cours}
\tableofcontents[hideallsubsections]
\end{frame}
% !TEX encoding = IsoLatin9

%%%%%%%%%%%%%%%%%%%%% SECTION 1
\section{Passage de param�tres au \texttt{main}}
\begin{frame}
  \begin{columns}
    \column{4.8cm}
    \tableofcontents[currentsection,hideothersubsections]
    \column{7cm}
    
  \end{columns}
\end{frame}
  
\begin{frame}[fragile]
\frametitle{La fonction \bvrb|main|}
\begin{itemize}
\setlength\itemsep{1em}
\item Un code C doit contenir obligatoirement une fonction \bvrb|main|.
\item Le \bvrb|main| est le point d'entr�e de l'ex�cutable.
\end{itemize}
\begin{block}{}
Il est possible de communiquer des informations � l'ex�cutable.
\end{block}

Exemple : le programme \Verb|gedit| est un ex�cutable, on peut lui passer
en argument le nom du fichier � ouvrir.

\begin{termblock}{Ouvre l'�diteur de texte avec un fichier vide :}
%\vspace{-.3cm}
\lstset{escapeinside={��}}
\lstset{basicstyle=\scriptsize}
\begin{lstlisting}
>> gedit
\end{lstlisting}
\vspace{-.3cm}
\end{termblock}

\begin{termblock}{Ouvre le fichier \Verb|hello.c| :}
%\vspace{-.3cm}
\lstset{escapeinside={��}}
\lstset{basicstyle=\scriptsize}
\begin{lstlisting}
>> gedit hello.c
\end{lstlisting}
\vspace{-.3cm}
\end{termblock}

\end{frame}

\begin{frame}[fragile]
\frametitle{Arguments du \bvrb|main|}
\begin{block}{}
Il est possible de r�cup�rer les arguments pass�s
� l'ex�cutable gr�ce aux arguments du main.
\end{block}
\begin{codeblock}{}
\vspace{-.3cm}
\lstset{escapeinside={��}}
%\lstset{basicstyle=\scriptsize}
\begin{codeC}
int main (int argc, char *argv[]){
\end{codeC}
\vspace{-.3cm}
\end{codeblock}

\begin{itemize}
\setlength\itemsep{1em}
\item \bvrb|argc :| contient le nombre d'arguments
pass�s � l'ex�cutable (nombre de mots dans la ligne
de commande).
\item \bvrb|argv :| tableau des arguments.\\
\red{Le premier
argument est le nom de l'ex�cutable.}
\end{itemize}
\end{frame}

\begin{frame}[fragile]
\frametitle{Comment �a marche ?}
\begin{itemize}
\item Dans le terminal :
\begin{termblock}{}
\vspace{-.3cm}
\lstset{escapeinside={��}}
\lstset{basicstyle=\scriptsize}
\begin{lstlisting}
>> monprog arg1 arg2
\end{lstlisting}
\vspace{-.3cm}
\end{termblock}

\item Dans le code :
\begin{codeblock}{}
\vspace{-.3cm}
\lstset{escapeinside={��}}
%\lstset{basicstyle=\scriptsize}
\begin{codeC}
int main (int argc, char *argv[]){
\end{codeC}
\vspace{-.3cm}
\end{codeblock}

\item En m�moire :
\begin{figure}
\begin{tikzpicture}[
auto,
node distance = 1em,
block/.style  = {rectangle, inner sep = 0pt},
]
\node (argc) [block] {\Verb|argc|};
\node (argc3) [block,right = of argc] {
  \begin{tabular}{|c|}
    \hline
    3 \\
    \hline
    \end{tabular}
    };
\node (argv) [block,below = of argc] {\Verb|argv|};
\node (argcp) [block,right = of argv] {
  \begin{tabular}{|c|}
    \hline
    \phantom{3} \\
    \hline
    \end{tabular}
    };
\node(argc0) [block,right= of argcp] {
 \begin{tabular}{|c|}
    \hline
    \Verb|argv[0]| \\
    \hline
    \end{tabular}
    };
\node(argc0v) [block,right = of argc0] {
\begin{tabular}{|c|c|c|c|c|c|c|c|}
    \hline
    m & o & n & p & r & o & g & \Verb|\0|\\
    \hline
    \end{tabular}
    };

\node(argc1) [block,below= of argc0,yshift = 0.8 em] {
 \begin{tabular}{|c|}
    \hline
    \Verb|argv[1]| \\
    \hline
    \end{tabular}
    };
\node(argc1v) [block,right = of argc1] {
\begin{tabular}{|c|c|c|c|c|}
    \hline
    a & r & g & 1 & \Verb|\0|\\
    \hline
    \end{tabular}
    };

\node(argc2) [block,below= of argc1,yshift = 0.8em] {
 \begin{tabular}{|c|}
    \hline
    \Verb|argv[2]| \\
    \hline
    \end{tabular}
    };
\node(argc2v) [block,right = of argc2] {
\begin{tabular}{|c|c|c|c|c|}
    \hline
    a & r & g & 2 & \Verb|\0|\\
    \hline
    \end{tabular}
    };

\draw [black,->] (argc0.east) -- (argc0v.west) ;
\draw [black,->] (argc1.east) -- (argc1v.west) ;
\draw [black,->] (argc2.east) -- (argc2v.west) ;
\draw [black,->] (argcp.east) -- (argc0.west) ;

\end{tikzpicture}
\end{figure}

\end{itemize}
\end{frame}

\begin{frame}[fragile]
\frametitle{Un exemple}

\begin{codeblock}{\Verb|monprog.c|}
\vspace{-.3cm}
\lstset{escapeinside={��}}
\lstset{basicstyle=\scriptsize}
\begin{codeC}
int main (int argc, char * argv[]) {
 int i;
 for (i=1;i<argc;i++)
 {
  printf("Argument %d : %s\n",i,argv[i]);
 }
}
\end{codeC}
\vspace{-.3cm}
\end{codeblock}

\begin{termblock}{Test d'ex�cution}
%\vspace{-.3cm}
\lstset{escapeinside={��}}
\lstset{basicstyle=\scriptsize}
\begin{lstlisting}
>> monprog fic1.txt 100
Argument 1 : fic1.txt
Argument 2 : 100
\end{lstlisting}
\vspace{-.3cm}
\end{termblock}
\end{frame}

\begin{frame}
\frametitle{Utilisation}
\begin{itemize}
\setlength\itemsep{1.5em}
\item Permet de transmettre des informations
du shell au programme.
\item Utile dans les scripts shell qui appellent
plusieurs ex�cutables.
\item Permet de passer certains param�tres
� vos ex�cutable (nom de fichiers, identifiants, etc.)
\end{itemize}
\end{frame}
% !TEX encoding = IsoLatin9

%%%%%%%%%%%%%%%%%%%%% SECTION 1
\section{La compilation}
\begin{frame}
  \begin{columns}
    \column{4.8cm}
    \tableofcontents[currentsection,hideothersubsections]
    \column{7cm}
    
  \end{columns}
  
\end{frame}

%%%%%%%%%%%%%%%%%%%%%%%%%%%%%%%%%%%%%%%%%%%
%                  FRAME 1                %
%%%%%%%%%%%%%%%%%%%%%%%%%%%%%%%%%%%%%%%%%%%

\begin{frame}[t,fragile]
\frametitle{Les 4 �tapes de la compilation}

%%%%%%%%%%%%%% ALL SLIDES %%%%%%%%%%%%%%%%%%
\vspace{-0.7cm}
\begin{figure}[t]
\centering
\begin{tikzpicture} [
  block/.style    = { rectangle, draw=blue, thick, 
                      fill=blue!20, text width=1.8cm, text centered,
                      rounded corners, minimum height=2em },
 ablock/.style    = { rectangle, draw=red, thick, 
                      fill=red!20, text width=1.8cm, text centered,
                      rounded corners, minimum height=2em },
 line/.style     = { draw, very thick, ->, shorten >=1pt },
 aline/.style     = { draw, color=red,very thick, ->, shorten >=1pt },
 label/.style    = {midway,anchor=north,yshift=-0.3cm, text width=2.4cm,text centered},
  node distance = 2.5cm,
]
\node <1-|handout:1-> (cfile) [block] {\Verb|Bonjour.c|};

\node <1,4-|handout:1,4-> (ifile) [block, right of = cfile] {\Verb|bonjour.i|};
\node <2-3|handout:2-3> (ifile) [ablock, right of = cfile] {\Verb|bonjour.i|};

\node <1-3,7-|handout:1-3,7->(sfile) [block, right of = ifile] {\Verb|bonjour.s|};
\node <4-6|handout:4-6> (sfile) [ablock, right of = ifile] {\Verb|bonjour.s|};

\node <1-6,8-|handout:1-6,8-> (ofile) [block, right of = sfile] {\Verb|bonjour.o|};
\node <7|handout:7> (ofile) [ablock, right of = sfile] {\Verb|bonjour.o|};

\node <1-7|handout:1-7> (efile) [block, right of = ofile] {\Verb|bonjour|};
\node <8|handout:8> (efile) [ablock, right of = ofile] {\Verb|bonjour|};

\path <1,4-|handout:1,4-> [line] (cfile) --  node[label]{preprocessing}(ifile) ;
\path <2-3|handout:2-3> [aline] (cfile) --  node[label]{preprocessing}(ifile) ;

\path <1-3,7-|handout:1-3,7-> [line] (ifile) -- node[label]{compilation}(sfile) ;
\path <4-6|handout:4-6> [aline] (ifile) -- node[label]{compilation}(sfile) ;

\path <1-6,8-|handout:1-6,8-> [line] (sfile) -- node[label]{assemblage}(ofile) ;
\path <7|handout:7> [aline] (sfile) -- node[label]{assemblage}(ofile) ;

\path <1-7|handout:1-7> [line] (ofile) -- node[label]{�dition des liens}(efile) ;
\path <8|handout:8> [aline] (ofile) -- node[label]{�dition des liens}(efile) ;

\end{tikzpicture}
\end{figure}
%%%%%%%%%%%%%% SLIDE 1 %%%%%%%%%%%%%%%%%%
\begin{overlayarea}{\textwidth}{5cm}


\begin{onlyenv}<1|handout:1>
\vspace{-0.9cm}
\begin{columns}[t]
\column{.33\textwidth}
\begin{codeblock}{\Verb|bonjour.c|\hfill\tikz[remember picture,baseline=-.5ex] \coordinate(b1);}
\vspace{-.3cm}
\tikz[remember picture,baseline=0ex] (tab) {X};
\lstset{escapeinside={��}}
\lstset{basicstyle=\scriptsize}
\begin{codeC}
#include <stdio.h>

int main () {
 // Affiche "bonjour"
 printf("bonjour\n");
}
\end{codeC}
\end{codeblock}

\column{.38\textwidth}
\begin{termblock}{\tikz[remember picture,baseline=-.5ex] \coordinate(b2);\Verb|Terminal|}
\vspace{-.3cm}
\lstset{escapeinside={��}}
\lstset{basicstyle=\scriptsize}
\begin{lstlisting}
�\textbf{>>}�gcc bonjour.c -o bonjour
\end{lstlisting}
\centering{\tikz[remember picture,baseline=-0.5ex] \coordinate(b2_south);}
\vspace{-.6cm}
\end{termblock}
\vspace{0.8cm}
\begin{block}{}
%\centering{\tikz[remember picture,baseline=-.5ex] \coordinate(b3);}\\
Fichier Ex�cutable \Verb|Bonjour|
\end{block}
\end{columns}

\begin{alertblock}{}
Par d�faut \Verb|gcc| effectue les 4 �tapes.
\end{alertblock}

\begin{tikzpicture}[remember picture,overlay]
\draw (b1) edge[->, very thick,shorten >=10pt, shorten <=10pt] (b2);
\draw ($(b2_south)+(0,0.6)$) edge[->, very thick,shorten >=10pt, shorten <=10pt] ++(0,-1.6);

\end{tikzpicture}

\end{onlyenv}

%%%%%%%%%%%%%% SLIDE 2 %%%%%%%%%%%%%%%%%
\begin{onlyenv}<2|handout:2>
\vspace{-0.9cm}

\begin{block}{Preprocessing}
Le pr�processeur effectue diff�rentes op�rations de substitution
et de supression dans le code :
\begin{itemize}
\item Supression des commentaires (\bvrb|//| et  \bvrb|/* */|)
qui sont utiles au programmeur, mais inutiles pour le processeur.\\
\item Inclusion des fichier \Verb|.h| dans le fichier \Verb|.c| 
(directive \bvrb|#include|). Ici, il permet de donner le prototype
de la fonction \bvrb|printf| (son format).\\
\item Traitemenet des directives de compilation qui commencent par
un caract�re \bvrb|#| (voir plus loin).\\
\end{itemize}

\end{block}
\end{onlyenv}

%%%%%%%%%%%%%% SLIDE 3 %%%%%%%%%%%%%%%%%%
\begin{onlyenv}<3|handout:3>
\vspace{-0.9cm}
\begin{columns}[t]
\column{.33\textwidth}
\begin{codeblock}{\Verb|bonjour.c|\hfill\tikz[remember picture,baseline=-.5ex] \coordinate(b1_2);}
\vspace{-.3cm}
\lstset{escapeinside={��}}
\lstset{basicstyle=\scriptsize}
\begin{codeC}
#include <stdio.h>

int main () {
 // Affiche "bonjour"
 printf("bonjour\n");
}
\end{codeC}
\end{codeblock}

\column{.6\textwidth}
\begin{termblock}{\tikz[remember picture,baseline=-.5ex] \coordinate(b2_2);\Verb|Terminal|}
\vspace{-.3cm}
\lstset{escapeinside={��}}
\lstset{basicstyle=\scriptsize}
\begin{lstlisting}
�\textbf{>>}�gcc -E Bonjour.c -o Bonjour.i
\end{lstlisting}
\centering{\tikz[remember picture,baseline=-0.5ex] \coordinate(b2_2_south);}
\vspace{-.6cm}
\end{termblock}
\vspace{0.2cm}
\begin{codeblock}{\Verb|bonjour.i|}
\vspace{-.3cm}
\lstset{escapeinside={��}}
\lstset{basicstyle=\scriptsize}
\begin{codeC}
(...) extern int printf 
 (__const char *__restrict__format, ...);
(...)
# 2 "bonjour.c" 2
int main () {
 printf("bonjour\n");
}
\end{codeC}
\end{codeblock}
\end{columns}

\begin{tikzpicture}[remember picture,overlay]
\draw (b1_2) edge[->, very thick,shorten >=3pt, shorten <=2pt] (b2_2);
\draw ($(b2_2_south)+(0,0.6)$) edge[->, very thick , shorten <=10pt] ++(0,-0.7);

\end{tikzpicture}

\end{onlyenv}


%%%%%%%%%%%%%% SLIDE 4 %%%%%%%%%%%%%%%%%%
\begin{onlyenv}<4|handout:4>
\begin{block}{La compilation}
La compilation (au sens strict) tranforme le langage C en assembleur.
\end{block}
\end{onlyenv}



%%%%%%%%%%%%%% SLIDE 5 %%%%%%%%%%%%%%%%%%
\begin{onlyenv}<5|handout:5>

\vspace{-0.9cm}
\begin{columns}[t]
\column{.33\textwidth}
\begin{codeblock}{\Verb|bonjour.c|\hfill\tikz[remember picture,baseline=-.5ex] \coordinate(b1_2);}
\vspace{-.3cm}
\lstset{escapeinside={��}}
\lstset{basicstyle=\scriptsize}
\begin{codeC}
#include <stdio.h>

int main () {
 // Affiche "bonjour"
 printf("bonjour\n");
}
\end{codeC}
\end{codeblock}

\column{.6\textwidth}
\begin{termblock}{\tikz[remember picture,baseline=-.5ex] \coordinate(b2_2);\Verb|Terminal|}
\vspace{-.3cm}
\lstset{escapeinside={��}}
\lstset{basicstyle=\scriptsize}
\begin{lstlisting}
�\textbf{>>}�gcc -S Bonjour.c -o bonjour.s
\end{lstlisting}
\centering{\tikz[remember picture,baseline=-0.5ex] \coordinate(b2_2_south);}
\vspace{-.6cm}
\end{termblock}

\vspace{0.2cm}

\begin{codeblock}{\Verb|bonjour.s|}
\vspace{-.3cm}
\lstset{escapeinside={��}}
\lstset{basicstyle=\scriptsize}
\begin{codeC}
	.file	"bonjour.c"
	.section	.rodata
.LC0:
	.string	"bonjour"
(...)
	movl	$.LC0, %edi
	call	puts
(...)
	.section	.note.GNU-stack,"",@progbits
\end{codeC}
%$
\end{codeblock}
\end{columns}

\begin{tikzpicture}[remember picture,overlay]
\draw (b1_2) edge[->, very thick,shorten >=3pt, shorten <=2pt] (b2_2);
\draw ($(b2_2_south)+(0,0.6)$) edge[->, very thick , shorten <=10pt] ++(0,-0.7);

\end{tikzpicture}

\end{onlyenv}


%%%%%%%%%%%%%% SLIDE 6 %%%%%%%%%%%%%%%%%%
\begin{onlyenv}<6|handout:6>
\vspace{-3.9cm}
\begin{codeblock}{\Verb|bonjour.s| complet !}
\vspace{-.3cm}
\lstset{escapeinside={��}}
\lstset{basicstyle=\scriptsize}
\begin{codeC}
	.file	"bonjour.c"
	.section	.rodata
.LC0:
	.string	"bonjour"
	.text
	.globl	main
	.type	main, @function
main:
.LFB0:
	.cfi_startproc
	pushq	%rbp
	.cfi_def_cfa_offset 16
	.cfi_offset 6, -16
	movq	%rsp, %rbp
	.cfi_def_cfa_register 6
	movl	$.LC0, %edi
	call	puts
	popq	%rbp
	.cfi_def_cfa 7, 8
	ret
	.cfi_endproc
.LFE0:
	.size	main, .-main
	.ident	"GCC: (Ubuntu/Linaro 4.6.3-1ubuntu5) 4.6.3"
	.section	.note.GNU-stack,"",@progbits
\end{codeC}
%$
\vspace{-0.3cm}
\end{codeblock}

\end{onlyenv}


%%%%%%%%%%%%%% SLIDE 7 %%%%%%%%%%%%%%%%%%
\begin{onlyenv}<7|handout:7>

\vspace{-0.9cm}
\begin{block}{}
Le code assembleur (encore lisible) est transform�
en code machine binaire.
\end{block}
\begin{columns}[t]
\column{.33\textwidth}
\begin{codeblock}{\Verb|bonjour.s|\hfill\tikz[remember picture,baseline=-.5ex] \coordinate(b1_2);}
\vspace{-.3cm}
\lstset{escapeinside={��}}
\lstset{basicstyle=\scriptsize}
\begin{codeC}
	.file	"bonjour.c"
	.section	.rodata
.LC0:
	.string	"bonjour"
(...)
	movl	$.LC0, %edi
	call	puts
(...)
	.section	
 .note.GNU-stack,"",
 @progbits
\end{codeC}
%$
\vspace{-0.3cm}
\end{codeblock}

\column{.6\textwidth}
\begin{termblock}{\tikz[remember picture,baseline=-.5ex] \coordinate(b2_2);\Verb|Terminal|}
\vspace{-.3cm}
\lstset{escapeinside={��}}
\lstset{basicstyle=\scriptsize}
\begin{lstlisting}
�\textbf{>>}�gcc -c bonjour.s
�\textbf{>>}�od -x bonjour.o
\end{lstlisting}
\centering{\tikz[remember picture,baseline=-0.5ex] \coordinate(b2_2_south);}
\vspace{-.6cm}
\end{termblock}

\vspace{0.2cm}

\begin{termblock}{\Verb|bonjour.o|}
\vspace{-.3cm}
\lstset{escapeinside={��}}
\lstset{basicstyle=\tiny}
\begin{lstlisting}
0000000 457f 464c 0102 0001 0000 0000 0000 0000
0000020 0001 003e 0001 0000 0000 0000 0000 0000
0000040 0000 0000 0000 0000 0128 0000 0000 0000
...
\end{lstlisting}
\end{termblock}
\end{columns}

\begin{tikzpicture}[remember picture,overlay]
\draw (b1_2) edge[->, very thick,shorten >=3pt, shorten <=2pt] (b2_2);
\draw ($(b2_2_south)+(0,0.6)$) edge[->, very thick , shorten <=10pt] ++(0,-0.7);

\end{tikzpicture}

\end{onlyenv}

%%%%%%%%%%%%%% SLIDE 8 %%%%%%%%%%%%%%%%%%
\begin{onlyenv}<8|handout:8>

\vspace{-0.9cm}
\begin{block}{}
Le fichier \Verb|bonjour.o| est incomplet, il manque le code
correspondant aux fonctions des biblioth�ques (ici : la fonction
\bvrb|printf| de la biblioth�que \Verb|stdio.h|).
\end{block}
\begin{columns}[t]
\column{.33\textwidth}
\begin{termblock}{\Verb|bonjour.o|\hfill\tikz[remember picture,baseline=-.5ex] \coordinate(b1_2);}
\Verb|(code binaire)|
\end{termblock}


\column{.6\textwidth}
\begin{termblock}{\tikz[remember picture,baseline=-.5ex] \coordinate(b2_2);\Verb|Terminal|}
\vspace{-.3cm}
\lstset{escapeinside={��}}
%\lstset{basicstyle=\scriptsize}
\begin{lstlisting}
�\textbf{>>}�gcc bonjour.o -o bonjour
\end{lstlisting}
\centering{\tikz[remember picture,baseline=-0.5ex] \coordinate(b2_2_south);}
\vspace{-.6cm}
\end{termblock}

\vspace{0.2cm}

\begin{termblock}{\Verb|bonjour|}
Fichier ex�cutable
\end{termblock}

\end{columns}

\begin{tikzpicture}[remember picture,overlay]
\draw (b1_2) edge[->, very thick,shorten >=3pt, shorten <=2pt] (b2_2);
\draw ($(b2_2_south)+(0,0.6)$) edge[->, very thick , shorten <=10pt] ++(0,-0.7);

\end{tikzpicture}


\end{onlyenv}
\end{overlayarea}

\end{frame}

%%%%%%%%%%%%%%%%%%%%%%%%%%%%%%%%%%%%%%%%%%%
%                  FRAME 2                %
%%%%%%%%%%%%%%%%%%%%%%%%%%%%%%%%%%%%%%%%%%%
\begin{frame}[fragile]
\frametitle{La directive \bvrb|\#define|}
%\frametitle{L'instruction \bvrb|if...else|}


\begin{itemize}
\item D�claration de constantes (ou d'une expression fixe quelconque) :\\
\bvrb|#define �textit�identificateur reste_de_la_ligne�|\\
\item Lorsque le pr�processeur lit une ligne de ce type, il remplace
toutes les occurences suivantes de \bvrb|�textit�identificateur�| dans le fichier texte
par \bvrb|�textit�reste_de_la_ligne�|\\
\item Par convention, on �crit l'identificateur en MAJUSCULE.
\end{itemize}
\begin{columns}
\column{0.3\textwidth}
\begin{codeblock}{\Verb|exemple.c|\hfill\tikz[remember picture,baseline=-.5ex] \coordinate(lc);}
%\vspace{-.3cm}
\lstset{escapeinside={��}}
\lstset{basicstyle=\scriptsize}
\begin{codeC}
#define PI 3.14159

int main()
{
  int x;
  x = PI*2 ;
}
\end{codeC}
\end{codeblock}
\column{0.3\textwidth}

\begin{termblock}{\tikz[remember picture,baseline=-.5ex] \coordinate(li);\Verb|Terminal|}
\vspace{-.3cm}
\lstset{escapeinside={��}}
\lstset{basicstyle=\scriptsize}
\begin{lstlisting}
�\textbf{>>}� gcc -E exemple.c 
  > exemple.i
\end{lstlisting}
\end{termblock}
\column{0.3\textwidth}
\begin{codeblock}{\tikz[remember picture,baseline=-.5ex] \coordinate(li);\Verb|exemple.i|}
%\vspace{-.3cm}
\lstset{escapeinside={��}}
\lstset{basicstyle=\scriptsize}
\begin{codeC}
# 2 "exemple.c" 2

int main()
{
  int x;
  x = 3.14159*2 ;
}
\end{codeC}
\end{codeblock}
\end{columns}

\begin{tikzpicture}[remember picture,overlay]
\draw (lc) edge[->, very thick,shorten >=6pt, shorten <=6pt] (li);


\end{tikzpicture}

\end{frame}

\begin{frame}[fragile]
\frametitle{Conseils sur le \bvrb|\#define|}
\begin{itemize}
\item Il est fortement recommand� de placer les
\bvrb|#define| au d�but de votre programme au m�me endroit
que les \bvrb|#include|.\\
\item Limitez l'utilisation des \bvrb|#define| � des options de
compilation ou � des constantes valables et utilis�es dans tout le
programme.\\
\end{itemize}
\begin{columns}
\column{0.4\textwidth}
\begin{flushright}
Placer ici les \bvrb|#define| de vos programmes.
\tikz[remember picture,baseline=-.5ex]\coordinate(left);
\end{flushright}

\column{0.4\textwidth}
\begin{codeblock}{}
%\vspace{-.3cm}
\lstset{escapeinside={��}}
\lstset{basicstyle=\scriptsize}
\begin{codeC}
#include <stdio.h>
�\tikz[remember picture,baseline=-.5ex]\coordinate(d1);�#define PI 3.14159
�\tikz[remember picture,baseline=-.5ex]\coordinate(d2);�#define NMAX 500
�\tikz[remember picture,baseline=-.5ex]\coordinate(d3);�#define DEBUG_MODE

int main()
{
...
}
\end{codeC}
\end{codeblock}
\end{columns}


\begin{tikzpicture}[remember picture,overlay]
\draw (left) edge[->, gray, thick,shorten >=2pt, shorten <=2pt] (d1);
\draw (left) edge[->, gray, thick,shorten >=2pt, shorten <=2pt] (d2);
\draw (left) edge[->, gray, thick,shorten >=2pt, shorten <=2pt] (d3);
\end{tikzpicture}

\end{frame}

\begin{frame}[fragile]
\frametitle{Les directives pour le pr�processeur}
\begin{itemize}
\item \bvrb|#define �textit�identificateur reste_de_la_ligne�|\\
Remplace \bvrb|�textit�identificateur�| par  \bvrb|�textit�reste_de_la_ligne�|
jusqu'� la fin du programme ou jusqu'� une instruction \bvrb|#undef|.\\
\vspace{0.5cm}
\item \bvrb|#undef �textit�identificateur�|\\
Marque la fin du remplacement syst�matique initi� par \bvrb|#define|\\
\vspace{0.5cm}

\item \bvrb|#ifdef �textit�identificateur� ... #endif|\\
Inlus la partie de programme situ�e entre \bvrb|#ifdef| et \bvrb|#endif| si
l'identificateur a �t� d�clar� avant dans un \bvrb|#define|. Sinon, la partie de
programme concern�e est supprim�e avant compilation.\\
\end{itemize}

\end{frame}

\begin{frame}[fragile]
\frametitle{Exemple}

\begin{columns}

\column{.48\textwidth}
\begin{codeblock}{\centering{\Verb|exemple.c|\tikz[remember picture,baseline=-.5ex] \coordinate(lc);}}
\lstset{escapeinside={��}}
\lstset{basicstyle=\scriptsize}
\begin{codeC}
#define PI 3.14159

int main()
{
  int x;
  x = PI*2;
#ifdef PI
  printf("Constante PI d�finie");
#endif
  printf("x=%f",x);
#undef PI
#ifdef PI
  printf("Ce printf sera supprim�
par le pr�processeur");
#endif
}
\end{codeC}
\end{codeblock}

\column{.48\textwidth}
\begin{codeblock}{\centering{\tikz[remember picture,baseline=-.5ex] \coordinate(li);\Verb|exemple.i|}}
\lstset{escapeinside={��}}
\lstset{basicstyle=\scriptsize}
\begin{codeC}
#2 exemple.c 2

int main()
{
  int x;
  x = PI*2;

  printf("Constante PI d�finie");

  printf("x=%f",x);





}
\end{codeC}
\end{codeblock}

\end{columns}

\begin{tikzpicture}[remember picture,overlay]
\draw (lc) edge[->, very thick,shorten >=6pt, shorten <=6pt] (li);
\end{tikzpicture}


\end{frame}
% !TEX encoding = IsoLatin9

%%%%%%%%%%%%%%%%%%%%% SECTION 1
\section{Programmation modulaire}
\begin{frame}
  \begin{columns}
    \column{4.8cm}
    \tableofcontents[currentsection,hideothersubsections]
    \column{7cm}
    
  \end{columns}
\end{frame}
 
\begin{frame}
\frametitle{Etat des lieux}
\begin{itemize}
\setlength\itemsep{1.2em}
\item Vos programmes sont de plus en plus gros.
\item Certaines fonctionnalit�s ne sont pas sp�cifiques � un seul programme
(exemple fonctions d'E/S).
\item Le maintien et la compr�hension des programmes est difficile.
\item le travail collaboratif est presque impossible.
\item La compilation de tout le projet est n�cessaire � chaque modification (m�me minime).
\end{itemize}
\end{frame}

\begin{frame}
\frametitle{Une solution : la programmation modulaire}

\begin{block}{}
Possibilit� de r�partir un programme
sur \red{plusieurs fichiers} r�unissant les
fonctionnalit�s et d�finitions d'un aspect
particulier du programme.
\end{block}
Chacun de ces d�coupages est appel� 
\red{un module}.\\
Ex : 
\begin{itemize}
\item Un module d'E/S, 
\item Un module pour les calculs,
\item Un module contenant le \bvrb|main()|.
\end{itemize}
\vspace{1em}
$\rightarrow$ n�cessite une r�flexion sur le d�coupage de votre
code.
\end{frame}

\begin{frame}[fragile]
\frametitle{Que contient un module ?}
\begin{block}{}
Un module est compos� :
\begin{itemize}
\item Un fichier ent�te \bvrb|.h|
\item Un fichier source \bvrb|.c|
\end{itemize}
\end{block}
\begin{itemize}
\item Le fichier ent�te d�crit l'interface 
du module.
\item Le fichier source contient l'impl�mentation
(la d�finition) des fonctions.
\end{itemize}
\vspace{1em}
Le fichier ent�te est inclus par la directive :
\begin{codeblock}{}
\vspace{-.3cm}
\lstset{escapeinside={��}}
\lstset{basicstyle=\scriptsize}
\begin{codeC}
#include "entete.h"
\end{codeC}
\vspace{-.3cm}
\end{codeblock}
\end{frame}

\begin{frame}[fragile]
\frametitle{Exemple : Les matrices}
\begin{figure}
\centering
\begin{tikzpicture}[
auto,
block/.style = {rectangle, draw=black,
rounded corners, minimum height=2em ,
  text width = 3.7cm,
  text badly centered,},
node distance = 1.5cm,
line/.style = {draw, thick, color=black, ->}
]

\node(main)  [block]
{
  Module \textbf{matrices}\\
  \textit{le programme principal}
};

\node(center) [below = of main] {};
\node (oper) [block, left of = center, xshift = -2.5cm]
{
  Module \textbf{operations}\\
  \textit{Calcul sur les matrices}
};

\node (io) [block, right of = center, xshift = 2.5cm]
{
  Module \textbf{matricesio}\\
  \textit{Entr�es/sorties pour les matrices}
};

\node (com) [block, below = of center]
{
  Module \textbf{commun}\\
  \textit{Description de la structure }\Verb|matrice|
};


\begin{scope} [every path/.style=line]
\path<2-> (com) -- node [midway, right]{utile �} (main);
\path<2-> (com) -- node [midway, left]{utile �} (oper);
\path<2-> (com) -- node [midway, right]{utile �} (io);
\path<2-> (oper) -- node [midway, left]{utile �} (main);
\path<2-> (io) -- node [midway, right]{utile �} (main);

\end{scope}
\end{tikzpicture}
\end{figure}

\end{frame}

\begin{frame}[fragile]
\frametitle{Le fichier ent�te (header) .h}
Le fichier ent�te contient :
\begin{itemize}
\item Des directives \bvrb|#include|
\item Des directives \bvrb|#define|
\item Les prototypes des fonctions du module
utilisables par les autres modules
\item des \red{d�claration} de variables globales
\item Des mod�les de structure.
\end{itemize}

\begin{alertblock}{}
Eviter de \red{d�finir} des variables (initialisation, etc.)
\end{alertblock}

\end{frame}



\begin{frame}[fragile]
\frametitle{Module \Verb|commun|}
\begin{columns}

\column{.47\textwidth}
\begin{codeblock}{commun.h}
\vspace{-.3cm}
\lstset{escapeinside={��}}
\lstset{basicstyle=\scriptsize}
\begin{codeC}
/* Description de la structure 
matrice */
#ifndef COMMUN_H
#define COMMUN_H

struct matrice {
   int col;
   int lig;
   float** mat;
};

#endif
\end{codeC}
\vspace{-.3cm}
\end{codeblock}

\column{.47\textwidth}
\begin{codeblock}{pas de commun.c}
%\vspace{-.3cm}
\lstset{escapeinside={��}}
\lstset{basicstyle=\scriptsize}
(pas de fonctions)
\begin{codeC}

\end{codeC}
\vspace{-.3cm}
\end{codeblock}

\end{columns}
\end{frame}



\begin{frame}[fragile]
\frametitle{Module \Verb|matricesio|}
\begin{columns}

\column{.47\textwidth}
\begin{codeblock}{matricesio.h}
\vspace{-.3cm}
\lstset{escapeinside={��}}
\lstset{basicstyle=\scriptsize}
\begin{codeC}
/* Entr�es-sorties pour
 les matrices */
#include "commun.h"


struct matrice* saisir();
void 
afficher(struct matrice* mat);
struct matrice* mat_uni();

\end{codeC}
\vspace{-.3cm}
\end{codeblock}

\column{.47\textwidth}
\begin{codeblock}{matricesio.c}
\vspace{-.3cm}
\lstset{escapeinside={��}}
\lstset{basicstyle=\scriptsize}
\begin{codeC}
#include <stdio.h>
#include "matriceio.h"

struct matrice* saisir(){
...
}

void afficher(struct matrice* 
mat){
...
}

struct matrice* mat_uni(){
...
}
\end{codeC}
\vspace{-.3cm}
\end{codeblock}

\end{columns}
\end{frame}





\begin{frame}[fragile]
\frametitle{Module \Verb|operations|}

\begin{codeblock}{operations.h}
\vspace{-.3cm}
\lstset{escapeinside={��}}
\lstset{basicstyle=\scriptsize}
\begin{codeC}
/* calculs sur les matrices */
#include "commun.h"

struct matrice* add(struct matrice* m1, struct matrice* m2);
struct matrice* mul(struct matrice* m1, struct matrice* m2);
struct matrice* mul_scal(struct matrice* m, float mu);
\end{codeC}
\vspace{-.3cm}
\end{codeblock}


\begin{codeblock}{operations.c}
\vspace{-.3cm}
\lstset{escapeinside={��}}
\lstset{basicstyle=\scriptsize}
\begin{codeC}
#include "operations.h"

struct matrice* add(struct matrice* m1, struct matrice* m2){
...
}

struct matrice* mul(struct matrice* m1, struct matrice* m2){
...
}

struct matrice* mul_scal(struct matrice* m, float mu){
...
}
\end{codeC}
\vspace{-.3cm}
\end{codeblock}

\end{frame}




\begin{frame}[fragile]
\frametitle{Module \Verb|matrices|}
\framesubtitle{Programme principal}
\begin{columns}

\column{.67\textwidth}
\begin{codeblock}{matrices.c}
\vspace{-.3cm}
\lstset{escapeinside={��}}
\lstset{basicstyle=\scriptsize}
\begin{codeC}
#include "commun.h"
#include "matricesio.h"
#include "operations.h"

int main(){
 struct matrice *m1, *m2, *m3, *m4, *m5, *m6;
 m1 = saisir();
 m2 = saisir();
 m3 = mat_uni();  
 m4 = add(m1,m2);
 afficher(m4);
 m5 = mul(m2,m3);
 afficher(m5);
 m6 = mul_scal(m4,4);
 afficher(m6);
 return 0;
}
\end{codeC}
\vspace{-.3cm}
\end{codeblock}

\column{.27\textwidth}
\begin{codeblock}{matrices.h}
%\vspace{-.3cm}
\lstset{escapeinside={��}}
\lstset{basicstyle=\scriptsize}
Pas de matrices.h
\begin{codeC}
\end{codeC}
\vspace{-.3cm}
\end{codeblock}

\end{columns}
\end{frame}



\begin{frame}[fragile]
\frametitle{Compilation s�par�e}
\begin{termblock}{}
\vspace{-.2cm}
\lstset{escapeinside={��}}
\lstset{basicstyle=\scriptsize}
\begin{Verbatim}[commandchars=���]
gcc -c matricesio.c
�vrbalert[2-]�gcc -c operations.c�
�vrbalert[3-]�gcc -c matrices.c�
�vrbalert[4-]�gcc matrices.o matricesio.o operations.o -o matrices�
\end{Verbatim}
\vspace{-.3cm}
\end{termblock}
\begin{figure}
\centering
\begin{tikzpicture}[
block/.style = {rectangle, draw=black,
  minimum height=2em ,
  text width = 1.6cm,
  text badly centered,
  font = \footnotesize,},
line/.style = {draw, thick, color=black, ->},
rline/.style = {draw, very thick, color=red, ->},
node distance = 0.6cm,
]
\node (c) [block] {\Verb|matrices.c|};
\node (i) [block, right = of c] {} ;
\node (a) [block, right =of i] {} ;
\node (o) [block, text width = 1.9cm, right =of a] {\Verb|matrices.o|} ;
\visible<3->{\node (om) [block, text width = 1.9cm,above = of o] {\Verb|matricesio.o|};}
\visible<2->{\node (oo) [block, text width = 1.9cm,above = of om] {\Verb|operations.o|};}
\visible<4->{\node (exe) [block, right = of om] {Ex�cutable \Verb|matrices|};}

\begin{scope} [every path/.style=line]
\path (c) -- node [midway, below, yshift = -0.3cm,font = \scriptsize] {preprocessing} (i) ;
\path (i) -- node [midway, below, yshift = -0.3cm,font = \scriptsize] {compilation} (a) ;
\path (a) -- node [midway, below, yshift = -0.3cm,font = \scriptsize] {compilation} (o) ;
\end{scope}
\begin{scope} [every path/.style=rline]
\path<4-> (o) -- node[midway, below, anchor = west, font = \footnotesize] {Edition des liens} (exe) ;
\path<4-> (om) -- (exe) ;
\path<4-> (oo) -- (exe) ;
\end{scope}


\end{tikzpicture}

\end{figure}

\end{frame}

\begin{frame}[fragile]
\frametitle{Probl�me des inclusions multiples}

\begin{figure}
\centering
\begin{tikzpicture}[
auto,
block/.style = {rectangle, draw=black,
rounded corners, minimum height=2em ,
  text width = 3.7cm,
  text badly centered,},
node distance = 1cm,
line/.style = {draw, thick, color=black, ->}
]

\node(main)  [block, text width = 4.5cm]
{
  \textbf{matrices.c}\\
\Verb|#include "matricesio.h"|\\
\Verb|#include "operations.h"|\\
\Verb|#include "commun.h"|
};

\node(center) [below = of main] {};
\node (oper) [block, left of = center, xshift = -2cm]
{
  \textbf{operations.h}\\
\Verb|#include "commun.h"|
};

\node (io) [block, right of = center, xshift = 2cm]
{
  \textbf{matricesio.h}\\
\Verb|#include "commun.h"|
};

\node (com) [block, below = of center]
{
  \textbf{commun.h}\\
 \Verb|struct matrice {|
};


\begin{scope} [every path/.style=line]
\path (com) -- (main);
\path (com) --  (oper);
\path (com) --  (io);
\path (oper) -- (main);
\path (io) -- (main);

\end{scope}
\end{tikzpicture}
\end{figure}
La structure matrice est d�finie 3 fois !

Risque d'erreur : \Verb|previous declaration of matrice|


\end{frame}
% !TEX encoding = IsoLatin9

%%%%%%%%%%%%%%%%%%%%% SECTION 1
\section{Makefile}
\begin{frame}
  \begin{columns}
    \column{4.8cm}
    \tableofcontents[currentsection,hideothersubsections]
    \column{7cm}
    
  \end{columns}
\end{frame}


\begin{frame}[fragile]
\frametitle{L'outil \Verb|make|}
\begin{block}{}
L'outil \Verb|make| est un programme pr�sent sous Linux permettant
de d�clencher un certain nombre de commandes \Verb|shell| � partir d'un
fichier nomm� \Verb|Makefile| ou \Verb|makefile|.
\end{block}
\Verb|make| permet dans votre cas d'automatiser la compilation en 
exploitant les avantages de la compilation s�par�e.

Proc�dez comme suit : 
\begin{enumerate}
\item D�finir les r�gles de compilation dans le fichier \Verb|Makefile|
\item Au moment de la compilation, invocation en entrant dans le terminal : \Verb|make|
\item La compilation se fait, en recompilant uniquement les fichiers n�cessaires (ceux qui ont �t�
modifi�s depuis la derni�re compilation, que l'on rep�re grace � la date de modification des fichiers).
\end{enumerate}

\end{frame}

\begin{frame}[fragile]
\frametitle{Structure du \Verb|Makefile|}
Un makefile est un ensemble de r�gles d�finies par :
\begin{codeblock}{}
\vspace{-.3cm}
\lstset{escapeinside={��}}
\lstset{basicstyle=\scriptsize}
\lstset{moredelim=**[is][\color{red}]{@}{@}}
\begin{codeC}
nom_cible : liste_dependances
<TAB>commandes
\end{codeC}
\vspace{-.3cm}
\end{codeblock}

\begin{itemize}
\item \Verb|nom_cible| : nom du fichier � g�n�rer
\item \Verb|liste_dependances| : liste des fichiers
permettant la g�n�ration de la cible
\item \Verb|commandes| (\red{Obligatoirement pr�c�d� d'une tabulation}) :
Commande de compilation permettant de g�n�rer la cible.
\end{itemize}

\end{frame}

\begin{frame}[fragile]
\frametitle{Exemple}
\begin{codeblock}{Makefile (fichier disponible sur Moodle)}
\vspace{-.3cm}
\lstset{escapeinside={��}}
\lstset{basicstyle=\scriptsize}
\lstset{moredelim=**[is][\color{red}]{@}{@}}
\begin{codeC}
matrices : matrices.o matricesio.o operations.o 
<tab>gcc matrices.o matricesio.o operations.o -o matrices

matrices.o : matrices.c operations.h matricesio.h commun.h 
<tab>gcc -c matrices.c

matricesio.o : matricesio.c matricesio.h commun.h 
<tab>gcc -c matricesio.c

operations.o : operations.c operations.h commun.h 
<tab>gcc -c operations.c

clean :
<tab>rm -f *.o matrices
\end{codeC}
\vspace{-.3cm}
\end{codeblock}

\begin{termblock}{}
%\vspace{-.3cm}
\lstset{escapeinside={��}}
\lstset{basicstyle=\scriptsize}
\lstset{moredelim=**[is][\color{red}]{@}{@}}
\begin{lstlisting}
>> make clean
>> make matrices
\end{lstlisting}
\vspace{-.3cm}
\end{termblock}

\end{frame}


\begin{frame}[fragile]
\frametitle{Autre possibilit�s du makefile}
\begin{itemize}
\item D�finition de variables
\begin{codeblock}{}
\vspace{-.3cm}
\lstset{escapeinside={��}}
\lstset{basicstyle=\scriptsize}
\lstset{moredelim=**[is][\color{red}]{@}{@}}
\begin{codeC}
CC = gcc
LDFLAGS = -lm

EXEC = matrices
OBJ = matrices.o operations.o matricesio.o

$(EXEC) : $(OBJ)
<tab>$(CC) -o $(EXEC) $(LDFLAGS)
\end{codeC}
\vspace{-.3cm}
\end{codeblock}
\item Variables pr�-d�finie :
\begin{itemize}
\item \bvrb|$@| : la cible  
\item \bvrb|$<| : la premi�re d�pendance
\item \bvrb|$^| : toutes les d�pendances
\end{itemize}
\item r�gles g�n�riques
\begin{codeblock}{}
\vspace{-.3cm}
\lstset{escapeinside={��}}
\lstset{basicstyle=\scriptsize}
\lstset{moredelim=**[is][\color{red}]{@}{@}}
\begin{codeC}
%.o: %.c
<tab>$(CC) $(CCFLAGS) -o $@ -c $<
\end{codeC}
\vspace{-.3cm}
\end{codeblock}
\end{itemize}
\end{frame}

\begin{frame}[fragile]
\frametitle{Exemple de makefile plus param�trable}

\begin{codeblock}{makefile}
\vspace{-.3cm}
\lstset{escapeinside={��}}
\lstset{basicstyle=\tiny}
\lstset{moredelim=**[is][\color{red}]{@}{@}}
\begin{codeC}
# options de compilation
CC = gcc
CCFLAGS = -Wall
LIBSDIR = 
LDFLAGS = -lm

# fichiers du projet
SRC = matrices.c matricesio.c operations.c
OBJ = $(SRC:.c=.o)
EXEC = matrices

# r�gle initiale
all: $(EXEC)

# d�pendance des .h
matrices.o: operations.h matricesio.h commun.h
matricesio.o: matricesio.h commun.h
operations.o: operations.h commun.h

# r�gles de compilation
%.o: %.c
	$(CC) $(CCFLAGS) -o $@ -c $<
	
# r�gles d'�dition de liens
$(EXEC): $(OBJ)
	$(CC) -o $@ $^ $(LIBSDIR) $(LDFLAGS)

# r�gles suppl�mentaires
clean:
	rm -f  $(EXEC) *.o

\end{codeC}
\vspace{-.3cm}
\end{codeblock}

\end{frame}
% !TEX encoding = IsoLatin9

%%%%%%%%%%%%%%%%%%%%% SECTION 1
\section{Les fichiers}
\begin{frame}
  \begin{columns}
    \column{4.8cm}
    \tableofcontents[currentsection,hideothersubsections]
    \column{7cm}
    
  \end{columns}
\end{frame}


\begin{frame}
\frametitle{Les fichiers}

\begin{block}{}
Une entr�e/sortie correspond � un tranfert
d'information entre la m�moire de la machine et
un p�riph�rique (�cran, clavier, disque dur, ...)
\end{block}

\vspace{1cm}
L'information est trait�e sur formes de \red{blocs}
qui repr�sentent un volume de donn�es tranf�r�es
\begin{block}{}
Un fichier est une suite de blocs stock�s sur un disque.
\end{block}
\end{frame}

\begin{frame}
\frametitle{Types de fichiers}
\begin{itemize}
\setlength\itemsep{1em}
\item \red{Les fichiers binaires :} Ils contiennent
l'information brute telle qu'elle est cod�e dans la
m�moire centrale. Ils se pr�sentent comme une suite d'octets
mis bout � bout, ils ne sont pas lisibles par l'humain.
\item \red{Les fichiers textes : } Ils contiennt des
donn�es "traduites" pour �tre lisibles. Ils contiennent
des s�parateurs (espaces, tabulations, retours � la ligne)
\end{itemize}

\end{frame}

\begin{frame}
\frametitle{Types d'acc�s aux fichiers}
\begin{itemize}
\setlength\itemsep{1em}
\item \red{Acc�s s�quentiel :} On acc�de aux blocs
successivement de premier au dernier. Ainsi pour atteindre une donn�e, 
il faut lire toutes les donn�es pr�c�dentes.
\item \red{Acc�s direct :} On d�place la position de lecture vers la position
voulue avant de lire la donn�e.
\item \red{Acc�s index� :} Il existe des index qui "pointent" vers des zones sp�cifiques du fichier
(utilis� pour les gros fichiers, comme les bases de donn�es).
\end{itemize}
\end{frame}

\begin{frame}[fragile]
\frametitle{Les structures \bvrb|FILE| et \bvrb|FILE *|}
\bvrb|FILE| est une structure contenant plusieurs champs n�cessaires
aux entr�es/sorties :
\begin{itemize}
\item Un pointeur sur la m�moire tampon associ�e au fichier.
\item Un pointeur sur la position courante dans le fichier
\item Un mode d'acc�s au fichier (lecture, �criture, ...)
\item Un indicateur d'erreur et un indicateur de fin de fichier
\end{itemize}
\begin{block}{}
Les fonctions C ne manipulent \red{que} des pointeurs sur \bvrb|FILE| : \bvrb|FILE *|
\end{block}
\begin{codeblock}{}
\vspace{-.3cm}
\lstset{escapeinside={��}}
\lstset{basicstyle=\scriptsize}
\begin{codeC}
int main() {
  FILE *fid;
\end{codeC}
\vspace{-.3cm}
\end{codeblock}
\end{frame}

\begin{frame}[fragile]
\frametitle{Entr�es/Sorties standards}
3 variable d'entr�es-sorties de type \bvrb|FILE *| sont pr�f�finies dans le
langage C (biblioth�que \bvrb|stdio.h|) et sont toujours utilisables :
\begin{itemize}
\item \Verb|stdin| : fichier d'entr�e standard, c'est � dire le clavier.
\item \Verb|stdout| : fichier de sortie standard, c'est � dire l'�cran.
\item \Verb|stderr| : fichier d'erreurs, par d�faut, l'�cran.
\end{itemize}

\begin{alertblock}{}
Toutes les op�rations faites sur les fichiers peuvent
�tre faites sur les entr�es-sorties pr�d�finies.
\end{alertblock}

\end{frame}

\begin{frame}[fragile]
\frametitle{Liste des principales fonctions}
\begin{codeblock}{}
\vspace{-.3cm}
\lstset{escapeinside={��}}
\lstset{basicstyle=\scriptsize}
\begin{codeC}
int n,p, fend;
FILE *fid ;
\end{codeC}
\vspace{-.3cm}
\end{codeblock}

\begin{figure}
\footnotesize
\begin{tabular}{|l|l|l|}
\hline
\textbf{nom} & \textbf{description} & \textbf{exemple} \\
\hline
\Verb|fopen| & ouverture d'un fichier & \Verb |fid = fopen("fic.txt","r");| \\
\hline
\Verb|fclose| & fermeture d'un fichier & \Verb |fclose(fid);| \\
\hline
\Verb|fwrite| & �criture dans un fichier binaire & \Verb |fwrite (&n,sizeof(int),1,fid);| \\
\hline
\Verb|fprintf| & �criture dans un fichier texte & \Verb |fprintf(fid,"%d",n);| \\
\hline
\Verb|fread| & lecture dans un fichier binaire & \Verb |fread(&p,sizeof(int),1,fid);| \\
\hline
\Verb|fscanf| & lecture dans un fichier texte & \Verb |fscanf(fid,"%d",&p);| \\
\hline
\Verb|feof| & test de fin de fichier & \Verb |fend = foef(fid);| \\
\hline
\end{tabular}
\end{figure}

\begin{alertblock}{}
Ces fonctions appartiennent � la biblioth�que \bvrb|stdio.h|
\end{alertblock}

\end{frame}

\begin{frame}[fragile]
\frametitle{Ouverture/Fermeture}
\begin{itemize}
\item Ouverture avec \bvrb|fopen()| : \\
\bvrb|FILE * fopen(char �textit�nomFichier[]�, char �textit�modeOuverture[]�);|\\
$\rightarrow$ \bvrb|fopen()| retourne \Verb|NULL| si l'op�ration �choue.

\item Fermeture avec \bvrb|fclose()| : \\
\bvrb|fclose (FILE * �textit�fid�);|\\
$\rightarrow$ \bvrb|fclose()| retourne 0 si tout s'est bien d�roul� et \Verb|EOF| sinon.
\end{itemize}

\begin{codeblock}{}
\vspace{-.3cm}
\lstset{escapeinside={��}}
%\lstset{basicstyle=\scriptsize}
\begin{codeC}
int main() {
 char nom[]="fic.txt";
 FILE *fid;
 fid=fopen(nom,"r");
 fclose(fid);
}
\end{codeC}
\vspace{-.3cm}
\end{codeblock}

\end{frame}

\begin{frame}[fragile]
\frametitle{Modes d'ouverture}
\begin{codeblock}{}
\vspace{-.3cm}
\lstset{escapeinside={��}}
%\lstset{basicstyle=\scriptsize}
\begin{codeC}
//Ouverture en mode lecture
fid=fopen(nom,"r");
\end{codeC}
\vspace{-.3cm}
\end{codeblock}


\begin{itemize}
\item \textbf{"r"} : ouverture du fichier en lecture
\item \textbf{"w"} : ouverture du fichier en �criture.\\
S'il n'existe pas, alors cr�ation, sinon, destruction.
\item \textbf{"a"} : ouverture en ajout (�criture � la fin du fichier)\\
S'il n'existe pas, alors cr�ation, sinon positionnement � la fin
\item "r+" : ouverture en lecture et �criture\\
Positionnement au d�but du fichier.
\item "w+": ouverture en lecture et �criture\\
S'il n'existe pas, alors cr�ation, sinon, destruction
\item "a+": Ouverture en lecture et ajout\\
S'il n'existe pas, alors cr�ation. La lecture est positionn�e au d�but,
l'�criture � la fin.
\end{itemize}
\end{frame}

\begin{frame}[fragile]
\frametitle{Lecture/Ecriture dans un fichier binaire}
\begin{itemize}
\item Lecture avec \bvrb|fread()| :\\
\bvrb|int fread (type *adr, int taille, int nbloc, FILE *fid);|\\
$\rightarrow$ Place les donn�es lues dans \Verb|adr|.\\
$\rightarrow$ retourne le nombre de valeurs lues.

\item Ecriture avec \bvrb|fwrite()| :\\
\bvrb|int fwrite (type *adre, int taille, int nbloc, FILE *fid);|\\
$\rightarrow$ Ecrit les donn�es de \Verb|adr| dans le fichier.\\
$\rightarrow$ retourne le nombre de blocs �crits.


\item \bvrb|taille| : taille du bloc � �crire ou � lire (g�n�ralement d�termin� par \bvrb|sizeof()|).
\item \bvrb|nbloc| : nombre de bloc � �crire ou � lire (ex : dimension d'un tableau).
\end{itemize}
\end{frame}

\begin{frame}[fragile]
\frametitle{Lecture/Ecriture format�e dans un fichier texte}
\begin{itemize}
\item Lecture avec \bvrb|fscanf()| (comme \bvrb|scanf|) :\\
\bvrb|int fscanf (FILE *fid, format, ...);|
\item Ecriture avec \bvrb|fprintf()| (commet \bvrb|printf|) :\\
\bvrb|int fprintf (FILE *fid, format, ...);|
\item \bvrb|format| : M�me utilisation que dans scanf ou printf
\item Diff�rence avec le binaire : les fichiers sont lisibles par l'humain.
\end{itemize}
\end{frame}
\end{document}