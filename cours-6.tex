\ifdefined\ishandout
\documentclass[handout]{beamer}
\else
\documentclass{beamer}
\fi

\usepackage[frenchb]{babel}
\usepackage[T1]{fontenc}
\usepackage[latin1]{inputenc}
\usepackage{hyperref}
\usepackage{multirow}
\usepackage{listings}
\usepackage{fancyvrb}
\usepackage{tikz}
\usepackage{framed}
\usepackage{algorithm}
\usepackage{algorithmic}
\usepackage{xcolor}
\usepackage{color, colortbl}
\usepackage{handoutWithNotes}
\usepackage{amsmath}
\usetikzlibrary{shapes.geometric}
\usetikzlibrary{positioning}
\usetikzlibrary{shapes.arrows, chains}
\usetikzlibrary{arrows,calc}
\usetikzlibrary{shapes.multipart}
\usepackage{array}
\usetheme{Boadilla}
\definecolor{BlueGreen}{cmyk}{0.85,0,0.33,0}
\definecolor{Gray}{rgb}{0.8,0.8,0.8}

\ifdefined\ishandout
\pgfpagesuselayout{3 on 1 with notes}[a4paper,border shrink=5mm]
\usecolortheme{dove}
\else
\usecolortheme{dolphin}
\fi


\lstnewenvironment{codeC}
{ \lstset{language=C,
    otherkeywords={printf,scanf}}
}
{}

\ifdefined\ishandout
\definecolor{mygreen}{rgb}{0,0,0}
\definecolor{mymauve}{rgb}{0,0,0}
\definecolor{myblue}{rgb}{0,0,0}
\else
\definecolor{mygreen}{rgb}{0,0.6,0}
\definecolor{mymauve}{rgb}{0.58,0,0.82}
\definecolor{myblue}{rgb}{0,0,1}

\fi

\definecolor{mygray}{rgb}{0.5,0.5,0.5}


\lstset{language=C,
% breakatwhitespace=false,         % sets if automatic breaks should only happen at whitespace
%  breaklines=true,                 % sets automatic line breaking
%  captionpos=b,                
commentstyle=\itshape\color{mymauve},
keywordstyle=\bfseries\color{myblue},
%numbers=left,                    % where to put the line-numbers; possible values are (none, left, right)
%  numbersep=8pt,                   % how far the line-numbers are from the code
%  numberstyle=\tiny\color{mygray}, % the style that is used for the line-numbers
  rulecolor=\color{black},         % if not set, the frame-color may be changed on line-breaks within not-black text (e.g. comments (green here))
%  showspaces=false,                % show spaces everywhere adding particular underscores; it overrides 'showstringspaces'
  showstringspaces=false,          % underline spaces within strings only
%  showtabs=false,                  % show tabs within strings adding particular underscores
%  stepnumber=2,                    % the step between two line-numbers. If it's 1, each line will be numbered
  stringstyle=\color{mygreen},     % string literal style
%  tabsize=2 
}
\ifdefined\ishandout
\newcommand{\red}{\textbf}
\else
\newcommand{\red}{\textcolor{red}}
\fi
%\newcommand \emph
%Default size : 12.8 cm * 9.6 cm

\newcommand{\tmark}[1]{\tikz[remember picture, baseline=-.5ex]{\coordinate(#1);}}

\ifdefined\ishandout
\newenvironment<>{codeblock}[1]{%begin
  \setbeamercolor{block title}{fg=black,bg=lightgray!80}%
  \begin{block}{#1}}
  % \begin{codeC}}
  %  {\end{codeC}
{  
\end{block}}

\newenvironment<>{termblock}[1]{
    \setbeamercolor{block title}{fg=black,bg=lightgray!90}%
    \begin{block}{#1}
}
%     \begin{Verbatim}}
{%\end{Verbatim}
\end{block}
}

\definecolor{bluegreen}{RGB}{0,0,0}
%\definecolor{bluegreen}{rgb}{0,0.6,0.8}
\else

\newenvironment<>{codeblock}[1]{%begin
  \setbeamercolor{block title}{fg=darkgray,bg=yellow}%
  \begin{block}{#1}}
  % \begin{codeC}}
  %  {\end{codeC}
{  
\end{block}}

\newenvironment<>{termblock}[1]{
    \setbeamercolor{block title}{fg=white,bg=lightgray}%
    \begin{block}{#1}}
%     \begin{Verbatim}}
{%\end{Verbatim}
\end{block}
}

\definecolor{bluegreen}{RGB}{0,149,182}
%\definecolor{bluegreen}{rgb}{0,0.6,0.8}
\fi

%\newcommand{\output}[1]{
\setbeamertemplate{navigation symbols}{}
\newcommand{\bvrb}{\Verb[commandchars=���,formatcom=\color{bluegreen}]}
\newcommand{\footvrb}{\footnotesize\Verb}
\newcommand{\vrbalert}[2][]{\visible<#1>{#2}}
%%% Commande pour les listes/arbres
\newcommand{\mvide}{\nodepart{one} \nodepart{two}}
\newcommand{\tvide}{\nodepart{one} \nodepart{two} \nodepart{three}}

%%Fin des commandes pour les listes/arbres.


\newcommand<>{\case}[2]{%
\filldraw (#1,#2) rectangle (1+#1,1+#2);}
%%% Param�tres du cours (� r�gler)
%Num�ro du cours
\newcommand{\nb}{6}
\title[cours n�6]{Les structures}
\author[]{julien.brajard@upmc.fr}
\institute[Polytech'UPMC]{Polytech'UPMC}
\date{23 Octobre 2017}
\begin{document}

\begin{frame}
\titlepage
\centering{
\url{https://moodle-sciences.upmc.fr} (cours Informatique G�n�rale MAIN-ROB)}
\end{frame}

\begin{frame}
\frametitle{Plan du cours}
\tableofcontents[hideallsubsections]
\end{frame}
% !TEX encoding = IsoLatin9
\section{Structures de donn�es}
\subsection{Introduction}
\begin{frame}
  \begin{columns}
    \column{4.8cm}
    \tableofcontents[currentsection,hideothersubsections,currentsubsection]
    \column{7cm}
    \centering{
      \includegraphics[width=4cm]{fig/linus.jpeg}
      }

      \textit{"I will, in fact, claim that the difference between a bad 
programmer and a good one is whether he considers his code or 
his data structures more important. Bad programmers worry 
about the code. Good programmers worry about data structures 
and their relationships."}\\
      \small{
        \hfill Linus Torvalds (1969-)\\
               \hfill cr�ateur de Linux}
    
  \end{columns}
  \end{frame}

\begin{frame}
\frametitle{D�finition}
\begin{block}{}
Une structure est un type de donn�es compos� de plusieurs
el�ments de type quelconque appel�s champs ou membres.
\end{block}
Les structures permettent de regrouper des informations
de types distincts mais ayant un lien s�mentique fort pour
le programmeur.
\begin{itemize}
\item Nom, Pr�nom, Date, lieu de naissance, adresse $\rightarrow$ \textbf{Identit�}
\item Jour, Mois, Ann�e $\rightarrow$ \textbf{Date}
\item Abscisse, Ordonn�e $\rightarrow$ \textbf{Point}
\end{itemize}

\begin{exampleblock}{Notes}
\begin{itemize}
\item Les structures compl�tent la notion de tableau car il devient
possible de regrouper des �l�ments de types diff�rents.
\item Les fonctions en C peuvent renvoyer une structure.
\end{itemize}

\end{exampleblock}

\end{frame}

\begin{frame}[fragile]
\frametitle{D�claration et initialisation}
Un mod�le de structure se d�fini de la fa�on suivante :
\begin{columns}
\column{.25\textwidth}
\begin{figure}
\begin{tikzpicture} [
auto,
remember picture,
 block/.style    = { rectangle, draw=blue, black, 
                         text width=2.8cm, text centered,
                         font = \footnotesize,
                        rounded corners, minimum height=2em },
node distance=0.5em,
]
\node (etiq) [block] {Etiquette de structure};
\node (point) [block, below= of etiq.south, anchor = north] 
{D�claration et initialisation d'un pointeur sur \bvrb|struct point|};

\end{tikzpicture}
\end{figure}
\column{.44\textwidth}

\begin{codeblock}{}
\vspace{-.3cm}
\lstset{escapeinside={��}}
\lstset{basicstyle=\scriptsize}
\begin{codeC}
struct �\tmark{c_etiq}�point {
 float x;�\tmark{c_champ1}�
 float y;�\tmark{c_champ2}�
 char couleur[10];�\tmark{c_champ3}�
};

struct point p1={1,-2,"Vert"};�\tmark{c_val}�

�\tmark{c_point1}�struct point * Ptr_point;
�\tmark{c_point2}�Ptr_point=&p1;
\end{codeC}
\vspace{-.3cm}
\end{codeblock}

\column{.25\textwidth}
\begin{figure}
\begin{tikzpicture} [
auto,
remember picture,
 block/.style    = { rectangle, draw=blue, black, 
                         text width=2.8cm, text centered,
                         font = \footnotesize,
                        rounded corners, minimum height=2em },
node distance=0.5em,
]
\node (champ) [block] {Champs de la structure};
\node (val) [block, below= of champ.south, anchor = north] 
{D�claration et initialisation d'une variable \bvrb|struct point|};

\end{tikzpicture}
\end{figure}

\end{columns}
\vspace{1em}
\begin{itemize}
\setlength\itemsep{1em}
\item L'�tiquette de structure permet de nommer le mod�le
\item L'initialisation est analogue � celle des tableaux.
\end{itemize}

\begin{tikzpicture}[remember picture,overlay, auto,
 line/.style     = { draw, color=black, ->},
]
\begin{scope} [every path/.style=line, thick, shorten >=2pt]
\path (etiq) -- (c_etiq) ;
\path (point) -- (c_point1) ;
\path (point) -- (c_point2) ;
\path (champ) -- (c_champ1); 
\path (champ) -- (c_champ2); 
\path (champ) -- (c_champ3); 
\path (val) -- (c_val); 

\end{scope}

\end{tikzpicture}

\end{frame}

\begin{frame}[fragile]
\frametitle{Acc�s aux membres de la structure}
\begin{block}{}
Pour acc�der aux membres de la structure, on utilise
l'op�rateur \textbf{\Large{\bvrb|.|}} (point)
\end{block}
Syntaxe : \bvrb|�textit�Nomdevariable.membre�|\\
\begin{codeblock}{Exemple}
\vspace{-.3cm}
\lstset{escapeinside={��}}
\lstset{basicstyle=\scriptsize}
\begin{codeC}
struct point {
 float x;
 float y;
 char couleur[10];
};
int main() {
 struct point origine;
 origine.x=0;
 origine.y=0;
 strcpy(origine.couleur,"noir");
 printf("abscisse : %f\n",pt.x);
 printf("ordonn�e : %f\n",pt.y);
 printf ("couleur : %s\n",pt.couleur);
 return(0);
}
\end{codeC}
\vspace{-.3cm}
\end{codeblock}


\end{frame}

\begin{frame}[fragile]
\frametitle{Remarques}
\begin{itemize}
\setlength\itemsep{1em}
\item Les noms des champs sont locaux � la structure
\begin{columns}
\column{0.6\textwidth}
\begin{codeblock}{}
\vspace{-.3cm}
\lstset{escapeinside={��}}
\lstset{basicstyle=\scriptsize}
\begin{codeC}
#include <stdio.h>
struct point {
 float x;
 float y;
 char couleur[10];
};
int main() {
 struct point M={1.1,0,"rouge");
 float x=5.1;
 printf("x=%f\n",x);
 printf("M.x=%f\n",M.x);
 return(0);
}
\end{codeC}
\vspace{-.3cm}
\end{codeblock}

\column{0.29\textwidth}
\begin{termblock}{Test d'ex�cution}
%\vspace{-.3cm}
\lstset{escapeinside={��}}
\lstset{basicstyle=\scriptsize}
\begin{lstlisting}
x=...
M.x=...
\end{lstlisting}
\vspace{-.3cm}
\end{termblock}
\end{columns}

\item L'usage veut que les mod�les de structure
soient plac�s entre les directives pr�processeur
(\bvrb|#|) et les prototypes des fonctions.

\end{itemize}
\end{frame}

\begin{frame}[fragile]
\frametitle{Op�ration sur les structures}
\begin{itemize}
\item R�cup�ration d'adresse par \bvrb|&|
\item Acc�s au membres par \bvrb|.| (point)
\item Affectation globale pour des variables d'un m�me mod�le.
\begin{codeblock}{}
\vspace{-.3cm}
\lstset{escapeinside={��}}
%\lstset{basicstyle=\scriptsize}
\begin{codeC}
struct point M={1.1,0,"rouge");
struct point N;
N=M;
}
\end{codeC}
\vspace{-.3cm}
\end{codeblock}
Petit truc : cela permet de faire des copies de tableaux sans boucle.
Ici \Verb|M.couleur| est un tableau de caract�res et il est copi�
dans le tableau \Verb|N.couleur|.\\
\red{Attention:} Cette astuce ne fonctionne que pour les tableaux
statiques.
\end{itemize}

\end{frame}

\begin{frame}[fragile]
\frametitle{Structures et fonctions}
\begin{itemize}
\item Possibilit� de passer une strutcture en param�tre d'une fonction
\begin{codeblock}{}
\vspace{-.3cm}
\lstset{escapeinside={��}}
\lstset{basicstyle=\scriptsize}
\begin{codeC}
void affichePoint (struct point pt) {
   printf("abscisse : %f\n",pt.x);
   printf("ordonn�e : %f\n",pt.y);
   printf ("couleur : %s\n",pt.couleur);
}
\end{codeC}
\vspace{-.3cm}
\end{codeblock}
\item Possibilit� de retourner une structure
\begin{codeblock}{}
\vspace{-.3cm}
\lstset{escapeinside={��}}
\lstset{basicstyle=\scriptsize}
\begin{codeC}
struct point construirePoint (float x, float y, char couleur[]) {
 struct point pt;
 pt.x=x;
 pt.y=y;
 strcpy(pt.couleur,couleur);
 return pt;
}
\end{codeC}
\vspace{-.3cm}
\end{codeblock}
\end{itemize}
\end{frame}

\begin{frame}[fragile]
\frametitle{Comparaison de structure}
\begin{block}{}
Il n'existe pas d'op�rateur de comparaison global.\\
\red{Il faut comparer champ par champ}
\end{block}
\begin{codeblock}{}
\vspace{-.3cm}
\lstset{escapeinside={��}}
\lstset{basicstyle=\scriptsize}
\begin{codeC}
int comparepoint (struct point P1, struct point P2) {
 /* renvoie 1 si les points sont �gaux 0 sinon*/
 int comp=0;
 if (((P1.x==P2.x) && (P1.y==P2.y)) &&
  !strcmp(P1.couleur,P2.couleur)) {
  comp=1;
 }
 return (comp);
}

\end{codeC}
\vspace{-.3cm}
\end{codeblock}
\end{frame}

\begin{frame}[fragile]
\frametitle{Pointeurs sur une structure}
\begin{itemize}
\item Possibilit� de passer une structure par adresse
\begin{codeblock}{}
\vspace{-.3cm}
\lstset{escapeinside={��}}
\lstset{basicstyle=\scriptsize}
\begin{codeC}
void symetrie (struct point *pt) {
 (*pt).x= -(*pt).x;
 (*pt).y= -(*pt).
}
\end{codeC}
\vspace{-.3cm}
\end{codeblock}
\item Pour all�ger l'�criture, utilisation d'un
symbole sp�cial : \bvrb|->|
\begin{codeblock}{}
\vspace{-.3cm}
\lstset{escapeinside={��}}
\lstset{basicstyle=\scriptsize}
\begin{codeC}
void symetrie (struct point *pt) {
 pt->x= -pt->x;
 pt->y= -pt->y;
}
\end{codeC}
\vspace{-.3cm}
\end{codeblock}
\end{itemize}
\begin{exampleblock}{remarques}
\begin{itemize}
\item Il est souvent pr�f�rable de passer les structures par adresse.
\item Les pointeurs sur des structures sont tr�s utilis�s.
\end{itemize}
\end{exampleblock}

\end{frame}

\begin{frame}[fragile]
\frametitle{Tableaux de structures}
\begin{itemize}
\item Les tableaux de structures sont \red{tr�s} utilis�s.
\begin{codeblock}{}
\vspace{-.3cm}
\lstset{escapeinside={��}}
\lstset{basicstyle=\scriptsize}
\begin{codeC}
struct point Segm[2]={{0,0,"rouge"},{1,2.3,"vert"}};
float dx,dy ;  
dx=Segm[1].x - Segm[0].x;
dy=Segm[1].y - Segm[0].y;
\end{codeC}
\vspace{-.3cm}
\end{codeblock}

\item On pr�f�re g�n�ralement utiliser des tableaux
de pointeurs de structure.
\begin{codeblock}{}
\vspace{-.3cm}
\lstset{escapeinside={��}}
\lstset{basicstyle=\scriptsize}
\begin{codeC}
struct point M1={0,0,"rouge"};
struct point M2={1,2.3,"vert"};
struct point *Segm[2]={&M1,&M2};
\end{codeC}
\vspace{-.3cm}
\end{codeblock}

\end{itemize}
\end{frame}

\begin{frame}[fragile]
\frametitle{Allocation dynamique de m�moire}

La taille d'une structure est donn�e
par l'op�rateur \bvrb|sizeof()|
\begin{codeblock}{}
\vspace{-.3cm}
\lstset{escapeinside={��}}
\lstset{basicstyle=\scriptsize}
\begin{codeC}
struct point M1={0,0,"rouge"};
struct point M2={1,2.3,"vert"};
struct point *Segm;
Segm=(struct point *)malloc(2*sizeof(struct Point));
Segm[0]=M1;
Segm[1]=M2;
\end{codeC}
\vspace{-.3cm}
\end{codeblock}
\vspace{1em}
\red{Attention} : La taille d'une structure est diff�rente
de la somme des tailles des champs qui la compose.

\end{frame}

\begin{frame}[fragile]
\frametitle{Types synonymes : \bvrb|typedef|}
\begin{block}{}
La fonctionnalit� \bvrb|typedef| permet de d�finir des
types synonymes.
\end{block}
\begin{columns}

\column{0.45\textwidth}
\begin{codeblock}{}
\vspace{-.3cm}
\lstset{escapeinside={��}}
\lstset{basicstyle=\scriptsize}
\begin{codeC}
typedef struct point {
int x;
int y;
char couleur[10];
} Point;

typedef int * PtrEntier
\end{codeC}
\vspace{-.3cm}
\end{codeblock}

\column{0.45\textwidth}
\begin{codeblock}{}
\vspace{-.3cm}
\lstset{escapeinside={��}}
\lstset{basicstyle=\scriptsize}
\begin{codeC}
int main() {
  Point P={1,2,"vert"};
  PtrEntier pn;
\end{codeC}
\vspace{-.3cm}
\end{codeblock}


�quivalent � :

\begin{codeblock}{}
\vspace{-.3cm}
\lstset{escapeinside={��}}
\lstset{basicstyle=\scriptsize}
\begin{codeC}
int main() {
struct point P={1,2,"vert"};
int * pn;
\end{codeC}
\vspace{-.3cm}
\end{codeblock}

\end{columns}

\begin{itemize}
\item Le type synonyme peut �tre utilis�
dans toutes les expressions (en particulier
les conversions \bvrb|()| et \bvrb|sizeof|).
\item Il rend les noms des types plus courts et intuitifs.
\item L'instruction \bvrb|typedef| est plac�e aux m�mes
endroits du code que les mod�les de structures et
commencent par une majuscule (convention).
\end{itemize}

\end{frame}

\begin{frame}
\frametitle{Que peut-on d�clarer comme champ de structure ?}
\begin{itemize}
\setlength\itemsep{1em}
\item Les types classiques (\bvrb|char|, \bvrb|int|, \bvrb|float|, ...) ;
\item Les tableaux statiques ;
\item Les pointeurs ;
\item D'autres structures ;
\item Des pointeurs sur une structure (\red{y compris elle-m�me}).
Ce sont alors des structures autor�f�rentielles ou r�cursives.
\end{itemize}
\end{frame}
% !TEX encoding = IsoLatin9
\subsection{Autres variables "structur�es"}
\begin{frame}
  \begin{columns}
    \column{4.8cm}
    \tableofcontents[currentsection,hideothersubsections,currentsubsection]
    \column{7cm}
   
  \end{columns}
  \end{frame}

\begin{frame}[fragile]
\frametitle{Les �num�rations}
\begin{block}{}
Les �num�rations permettent de
d�finir des constantes.

Elles accroient la lisibilit� des programmes.
\end{block}
D�claration de constante :
\begin{codeblock}{}
\vspace{-.3cm}
\lstset{escapeinside={��}}
\lstset{basicstyle=\scriptsize}
\begin{codeC}
enum {LUNDI,MARDI,MERCREDI,JEUDI,VENDREDI,SAMEDI,DIMANCHE};
\end{codeC}
\vspace{-.3cm}
\end{codeblock}
En fait, les constantes sont des entiers (dans l'exemple :
\Verb|LUNDI==0|, \Verb|MARDI==1|, etc.)
\begin{codeblock}{Exemple d'utilisation}
\vspace{-.3cm}
\lstset{escapeinside={��}}
\lstset{basicstyle=\scriptsize}
\begin{codeC}
int j=LUNDI ;
if (j==LUNDI) printf ("%d, c'est le jour du cours d
'info\n",j);
\end{codeC}
\vspace{-.3cm}
\end{codeblock}

\begin{termblock}{Test d'ex�cution}
%\vspace{-.3cm}
\lstset{escapeinside={��}}
\lstset{basicstyle=\scriptsize}
\begin{Verbatim}
0, c'est le jour du cours d'info
\end{Verbatim}
\vspace{-.3cm}
\end{termblock}

\end{frame}

\begin{frame}[fragile]
\frametitle{Une variable de type \bvrb|enum|}
\begin{block}{}
Il est possible de d�finir un type \bvrb|enum|.
\end{block}
\begin{codeblock}{}
\vspace{-.3cm}
\lstset{escapeinside={��}}
\lstset{basicstyle=\scriptsize}
\begin{codeC}
enum jour {LUNDI,MARDI,MERCREDI,JEUDI,VENDREDI,SAMEDI,DIMANCHE};
\end{codeC}
\vspace{-.3cm}
\end{codeblock}

\begin{codeblock}{Exemple d'utilisation}
\vspace{-.3cm}
\lstset{escapeinside={��}}
\lstset{basicstyle=\scriptsize}
\begin{codeC}
enum jour j1, j2 ;
j1 = LUNDI ;
j2 = MARDI ;
\end{codeC}
\vspace{-.3cm}
\end{codeblock}

\end{frame}

\begin{frame}[fragile]
\frametitle{Les unions}
\begin{block}{}
Les unions permettent de d�finir une variable
qui a un type "variable" parmi plusieurs.
\end{block}
La syntaxe est tr�s proche de celle utilis�e pour
les structures :
\begin{codeblock}{}
\vspace{-.3cm}
\lstset{escapeinside={��}}
\lstset{basicstyle=\scriptsize}
\begin{codeC}
union MonUnion {
 int entier;
 double reel;
 char chaine[100];
}
\end{codeC}
\vspace{-.3cm}
\end{codeblock}

Utilisation (comme pour les structures) :
\begin{codeblock}{}
\vspace{-.3cm}
\lstset{escapeinside={��}}
\lstset{basicstyle=\scriptsize}
\begin{codeC}
union MonUnion Var ;
Var.entier = 2 ;
\end{codeC}
\vspace{-.3cm}
\end{codeblock}

\begin{alertblock}{}
La taille de la variable est �gale � la taille du 
type le plus grand (dans l'exemple c'est \Verb|chaine| = 100 octets).
\end{alertblock}

\end{frame}

\begin{frame}[fragile]
\frametitle{Les unions}
Tous les champs de la variable de type \bvrb|union| partage
le m�me espace m�moire
\begin{figure}
\begin{tikzpicture} [
auto,
remember picture,
 block/.style    = { rectangle, draw=blue, black, 
                         text width=2.8cm, text centered,
                         font = \footnotesize,
                        rounded corners, minimum height=2em },
node distance=.5em,
]
\draw (0,0) node (var) [anchor = west] {Var};
\draw (0,-0.6) node (double) 
[anchor = west, rectangle, minimum height = 1.5em, minimum width = 4cm, inner sep = 0pt, fill = red, red]{};

\draw [very thick, red] 
($(double.west) + (0,-1.2)$) --
node [midway,below] {\Verb|double|}
 ($(double.east) + (0,-1.2)$) ;

\draw (0,-0.6) node (int) 
[anchor = west,rectangle, minimum height = 1.5em, minimum width = 2cm, inner sep = 0pt, fill = cyan, cyan]{};

\draw [very thick, cyan] 
($(int.west) + (0,-0.6)$) -- 
node [midway,below] {\Verb|int|}
($(int.east) + (0,-0.6)$) ;


\draw (0,-0.6) node (char) 
[anchor = west,rectangle, minimum height = 1.5em, minimum width = 10cm, inner sep = 0pt, draw=black]{};
\draw [very thick, black] 
($(char.west) + (0,-1.8)$) --
node [midway,below] {\Verb|char[100]|}
 ($(char.east) + (0,-1.8)$) ;


\end{tikzpicture}
\end{figure}

\begin{columns}
\column{0.6\textwidth}
\begin{codeblock}{}
\vspace{-.3cm}
\lstset{escapeinside={��}}
\lstset{basicstyle=\scriptsize}
\begin{codeC}
union MonUnion Var;
Var.entier = 200;
printf("Val int = %d\n",Var.entier);
Var.reel = 1200.05;
printf("Val double = %lf\n",Var.reel);
printf("Val int = %d\n",Var.entier);
\end{codeC}
\vspace{-.3cm}
\end{codeblock}

\column{0.33\textwidth}
\begin{termblock}{Test d'ex�cution}
%\vspace{-.3cm}
\lstset{escapeinside={��}}
\lstset{basicstyle=\scriptsize}
\begin{Verbatim}
Val int = 200
Val double = 1200.05
Val int = 858993459
\end{Verbatim}
\vspace{-.3cm}
\end{termblock}
\end{columns}


\end{frame}

\begin{frame}[fragile]
\frametitle{Adrese de la variable}
\begin{codeblock}{}
\vspace{-.3cm}
\lstset{escapeinside={��}}
\lstset{basicstyle=\scriptsize}
\begin{codeC}
union MonUnion variable;
printf("Adresse de l'union = %p\n",&variable);
printf("Adresse de la partie enti�re = %p\n",&variable.entier);
printf("Adresse de la partie r�elle = %p\n",&variable.reel);
\end{codeC}
\vspace{-.3cm}
\end{codeblock}

\begin{termblock}{Test d'ex�cution}
%\vspace{-.3cm}
\lstset{escapeinside={��}}
\lstset{basicstyle=\scriptsize}
\begin{Verbatim}
Adresse de l'union = 0023FF70
Adresse de la partie enti�re = 0023FF70
Adresse de la partie r�elle = 0023FF70
\end{Verbatim}
\vspace{-.3cm}
\end{termblock}

\end{frame}

\begin{frame}[fragile]



\begin{codeblock}{Mod�le de structure}
\vspace{-.3cm}
\lstset{escapeinside={��}}
\lstset{basicstyle=\scriptsize}
\begin{codeC}
enum type_t {ENTIER,CARACT};
struct touche {
  enum type_t type;
  union {
   int ent;
   char car;
  }
};
\end{codeC}
\vspace{-.3cm}
\end{codeblock}

\begin{codeblock}{Dans le programme :}
\vspace{-.3cm}
\lstset{escapeinside={��}}
\lstset{basicstyle=\scriptsize}
\begin{codeC}
struct touche t;
int n,res;
res=scanf("%d",&n);
if (res==0) {
  scanf("%c",&t.car);
  t.type=CARACT;
} 
else {
  t.ent=n;
  t.type=ENTIER;
}
\end{codeC}
\vspace{-.3cm}
\end{codeblock}


\end{frame}
% !TEX encoding = IsoLatin9
\subsection{Listes et arbres}
\begin{frame}
  \begin{columns}
    \column{4.8cm}
    \tableofcontents[currentsection,hideothersubsections,currentsubsection]
    \column{7cm}
   
  \end{columns}
  \end{frame}

\begin{frame}
\frametitle{Quelques d�finitions}

\begin{block}{Structure r�cursive}
Structure de donn�es dans laquelle un membre est un pointeur
vers une variable de la m�me structure.
\begin{itemize}
\item Donn�es �parpill�es en m�moire reli�es par des pointeurs.
\end{itemize}
\end{block}

\begin{block}{Liste cha�n�e}
Un membre pointe vers la variable suivante de la liste.
\end{block}

\begin{block}{List doublement cha�n�e}
Un membre pointe vers la variable suivante et un autre vers
la variable pr�c�dente.
\end{block}

\begin{block}{arbre binaire}
Deux pointeurs pointent vers deux membres suivants (les fils).
\end{block}

\end{frame}

\begin{frame}[fragile]
\frametitle{Illustration}

La liste cha�n�e :

\begin{figure}
\begin{tikzpicture}[
auto,
data/.style = {
  text width = 1em,
  align=center,
  rectangle split,
  rectangle split parts = 2,
  draw,
  text centered},
line/.style = {draw,thick, ->,},
node distance = 3em,
]
\node (tete) {tete};
\node[data, right of = tete,yshift=-0.7em] (1) {\mvide};
\node[data, right of = 1] (2) {\mvide};
\node[data, right of = 2] (3) {\mvide};
\node[data, right of = 3] (4) {\mvide};
\node[data, right of = 4] (5) {\mvide};
\node[right of = 5,yshift = -0.35em] (null) {\footnotesize\Verb|NULL|};

 \begin{scope}[every path/.style=line]
\path(tete) -- (1.one west) ;
\path(1.two east) -- (2.one west) ;
\path(2.two east) -- (3.one west) ;
\path(3.two east) -- (4.one west) ;
\path(4.two east) -- (5.one west) ;
\path(5.two east) -- (null.west);
\end{scope}

\end{tikzpicture}
\end{figure}

La liste doublement cha�n�e :

\begin{figure}
\begin{tikzpicture}[
auto,
data/.style = {
  text width = 1em,
  align=center,
  rectangle split,
  rectangle split parts = 3,
  draw,
  text centered},
line/.style = {draw=black,thick, ->,},
rline/.style = {draw=blue, thick, ->,}, 
node distance = 3em,
]
\node (tete) {tete};
\node[data, right of = tete,yshift=-0.7em] (1) {\tvide};
\node[data, right of = 1] (2) {\tvide};
\node[data, right of = 2] (3) {\tvide};
\node[data, right of = 3] (4) {\tvide};
\node[data, right of = 4] (5) {\tvide};
\node[right of = 5] (null) {\footnotesize\Verb|NULL|};
\node[yshift = -1.4em] (null2) {\footnotesize\Verb|NULL|};

 \begin{scope}[every path/.style=line]
\path(tete) -- (1.one west) ;
\path(1.two east) -- (2.one west) ;
\path(2.two east) -- (3.one west) ;
\path(3.two east) -- (4.one west) ;
\path(4.two east) -- (5.one west) ;
\path(5.two east) -- (null.west);
\end{scope}

\begin{scope}[every path/.style=rline]
\path(5.three west) -- (4.one east) ;
\path(4.three west) -- (3.one east) ;
\path(3.three west) -- (2.one east) ;
\path(2.three west) -- (1.one east) ;
\path(1.three west) -- (null2.east) ;
\end{scope}


\end{tikzpicture}
\end{figure}


L'arbre binaire :

\begin{figure}
\begin{tikzpicture}[
auto,
data/.style = {
  text width = 1em,
  align=center,
  rectangle split,
  rectangle split parts = 3,
  draw,
  text centered},
line/.style = {draw=black,thick, ->,},
node distance = 3em,
]
\node (tete) {tete};
\node[data, right of = tete,yshift=-0.7em] (1) {\tvide};
\node[data, right of = 1,yshift=-2em] (2) {\tvide};
\node[data, right of = 1,yshift=2em] (3) {\tvide};
\node[data, right of = 3,yshift=-1.2em] (4) {\tvide};
\node[data, right of = 3,yshift=1.2em] (5) {\tvide};

 \begin{scope}[every path/.style=line]
\path(tete) -- (1.one west) ;
\path(1.two east) -- (3.one west) ;
\path(1.three east) -- (2.one west) ;
\path(3.two east) -- (5.one west) ;
\path(3.three east) -- (4.one west) ;
\end{scope}


\end{tikzpicture}
\end{figure}

\end{frame}

\begin{frame}[fragile]
\frametitle{Mod�les}
\begin{codeblock}{Liste cha�n�e}
\vspace{-.3cm}
\lstset{escapeinside={��}}
\lstset{basicstyle=\scriptsize}
\begin{codeC}
struct ListeSimple{
 type_val val;
 struct ListeSimple *suivant;
};
\end{codeC}
\vspace{-.3cm}
\end{codeblock}

\begin{codeblock}{Liste doublement cha�n�e}
\vspace{-.3cm}
\lstset{escapeinside={��}}
\lstset{basicstyle=\scriptsize}
\begin{codeC}
struct ListeDouble{
 type_val val;
 struct ListeDouble *suivant;
 struct ListeDouble *precedent;
};
\end{codeC}
\vspace{-.3cm}
\end{codeblock}

\begin{codeblock}{Liste doublement cha�n�e}
\vspace{-.3cm}
\lstset{escapeinside={��}}
\lstset{basicstyle=\scriptsize}
\begin{codeC}
struct Arbre{
 type_val val;
 struct Arbre *fils_droit;
 struct Arbre *fils_gauche;
};
\end{codeC}
\vspace{-.3cm}
\end{codeblock}

\end{frame}

\begin{frame}
\frametitle{Utilisation}
\begin{block}{}
On �crit des fonctions sp�cifiques pour manipuler les listes :
\end{block}
\begin{itemize}
\setlength\itemsep{1em}
\item Cr�ation d'un maillon
\item Ajout d'un maillon
\item Supression d'un maillon
\item Traitement (affichage, recherche, concat�nation
\end{itemize}
\end{frame}

\begin{frame}[fragile]
\frametitle{Cr�ation d'un maillon}
\begin{codeblock}{Exemple de mod�le de liste}
\vspace{-.3cm}
\lstset{escapeinside={��}}
\lstset{basicstyle=\scriptsize}
\begin{codeC}
struct liste{
 int val;
 struct liste *suivant;
};

\end{codeC}
\vspace{-.3cm}
\end{codeblock}

\begin{codeblock}{Fonction creation}
\vspace{-.3cm}
\lstset{escapeinside={��}}
\lstset{basicstyle=\scriptsize}
\begin{codeC}
struct liste * creation() {
 struct liste * nouveau ;
 nouveau = (struct liste *) malloc (sizeof(struct liste));
 nouveau -> val = 0 ; //par exemple
 nouveau -> suivant = NULL ;
 return(nouveau) ;
\end{codeC}
\vspace{-.3cm}
\end{codeblock}

\end{frame}

\begin{frame}[fragile]
\frametitle{Supression}
Supprime l'�l�ment dont la valeur est �gale � \Verb|val|.
\begin{codeblock}{Fonction creation}
\vspace{-.3cm}
\lstset{escapeinside={��}}
\lstset{basicstyle=\scriptsize}
\begin{codeC}
struct liste *supprime(int val, struct liste *tete) {
 struct liste *maillon = tete ;
 struct liste *prec = tete ;
 while (maillon != NULL && maillon -> valeur != val) {
  prec = maillon ;
  maillon = maillon -> suivant ;
 }

 if (maillon != NULL) {
  if (maillon==tete)
   tete = prec -> suivant;
  else 
   prec -> suivant = maillon -> suivant;
  free(maillon);
 }
}
\end{codeC}
\vspace{-.3cm}
\end{codeblock}

\end{frame}

\begin{frame}[fragile]
\frametitle{Autre structure r�cursive : l'arbre}
\begin{block}{Vocabulaire}
\begin{description}
\item[Noeud de l'arbre] : El�ment de base de l'arbre qui contient l'information ;
\item[Fils d'un noeud N] : Noeud point� par le noeud N ;
\item[P�re d'un noeud F] : Noeud qui pointe sur F ;
\item[Racine de l'arbre] : Noeud (unique) qui n'a pas de p�re ;
\item[Feuille de l'arbre] : Noeud qui n'a aucun fils (pointeurs = \Verb|NULL|).
\end{description}

\end{block}

\begin{figure}
\begin{tikzpicture} [
  remember picture,
  auto,
  noeud/.style = {circle,draw=black,inner sep = 4pt, radius = 4pt},
  line/.style = {draw,->},
]

 \node [noeud, fill = black] (rac) {};
 \node [noeud, below of = rac, xshift = -2cm] (fg) {};
  \node [noeud, below of = rac, xshift = +2cm] (fd) {};

 \node [noeud, fill = green, below of = fg, xshift = -1.2cm] (fgg) {};
 \node [noeud, fill = green,below of = fg, xshift = 1.2cm] (fgd) {};

\node [noeud, fill = green,below of = fd, xshift = -1.2cm] (fdg) {};
 \node [noeud, fill = green,below of = fd, xshift = 1.2cm] (fdd) {};

\node [right of = rac] {racine};
\node [right of = fdd] {feuille};

 \begin{scope}[every path/.style=line]
\path (rac) -- (fg) ;
\path (rac) -- (fd) ;
\path (fg) -- (fgg) ;
\path (fg) -- (fgd) ;
\path (fd) -- (fdg) ;
\path (fd) -- (fdd) ;

\end{scope}

\end{tikzpicture}
\end{figure}

\end{frame}

\begin{frame}[fragile]
\frametitle{Deux types d'arbres}
\begin{block}{Arbre binaire}
Arbre dont chaque noeud a au plus deux fils
(qu'on d�signe fils gauche et fils droit)
\end{block}

\begin{block}{Arbre n-aire}
Arbre dont chaque noeud a un nombre ind�termin� de fils.
Les fils ont donc une structure de liste cha�n�e.
\end{block}

\begin{codeblock}{Mod�le de structure d'un arbre n-aire}
\vspace{-.3cm}
\lstset{escapeinside={��}}
\lstset{basicstyle=\scriptsize}
\begin{codeC}struct liste *supprime(int val, struct liste *tete) {
struct arbre {
  int valeur ;
  struct arbre *fils ;
  struct arbre *frere;
}
\end{codeC}
\vspace{-.3cm}
\end{codeblock}

\end{frame}
\end{document}