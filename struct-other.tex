% !TEX encoding = IsoLatin9
\subsection{Autres variables "structur�es"}
\begin{frame}
  \begin{columns}
    \column{4.8cm}
    \tableofcontents[currentsection,hideothersubsections,currentsubsection]
    \column{7cm}
   
  \end{columns}
  \end{frame}

\begin{frame}[fragile]
\frametitle{Les �num�rations}
\begin{block}{}
Les �num�rations permettent de
d�finir des constantes.

Elles accroient la lisibilit� des programmes.
\end{block}
D�claration de constante :
\begin{codeblock}{}
\vspace{-.3cm}
\lstset{escapeinside={��}}
\lstset{basicstyle=\scriptsize}
\begin{codeC}
enum {LUNDI,MARDI,MERCREDI,JEUDI,VENDREDI,SAMEDI,DIMANCHE};
\end{codeC}
\vspace{-.3cm}
\end{codeblock}
En fait, les constantes sont des entiers (dans l'exemple :
\Verb|LUNDI==0|, \Verb|MARDI==1|, etc.)
\begin{codeblock}{Exemple d'utilisation}
\vspace{-.3cm}
\lstset{escapeinside={��}}
\lstset{basicstyle=\scriptsize}
\begin{codeC}
int j=LUNDI ;
if (j==LUNDI) printf ("%d, c'est le jour du cours d
'info\n",j);
\end{codeC}
\vspace{-.3cm}
\end{codeblock}

\begin{termblock}{Test d'ex�cution}
%\vspace{-.3cm}
\lstset{escapeinside={��}}
\lstset{basicstyle=\scriptsize}
\begin{Verbatim}
0, c'est le jour du cours d'info
\end{Verbatim}
\vspace{-.3cm}
\end{termblock}

\end{frame}

\begin{frame}[fragile]
\frametitle{Une variable de type \bvrb|enum|}
\begin{block}{}
Il est possible de d�finir un type \bvrb|enum|.
\end{block}
\begin{codeblock}{}
\vspace{-.3cm}
\lstset{escapeinside={��}}
\lstset{basicstyle=\scriptsize}
\begin{codeC}
enum jour {LUNDI,MARDI,MERCREDI,JEUDI,VENDREDI,SAMEDI,DIMANCHE};
\end{codeC}
\vspace{-.3cm}
\end{codeblock}

\begin{codeblock}{Exemple d'utilisation}
\vspace{-.3cm}
\lstset{escapeinside={��}}
\lstset{basicstyle=\scriptsize}
\begin{codeC}
enum jour j1, j2 ;
j1 = LUNDI ;
j2 = MARDI ;
\end{codeC}
\vspace{-.3cm}
\end{codeblock}

\end{frame}

\begin{frame}[fragile]
\frametitle{Les unions}
\begin{block}{}
Les unions permettent de d�finir une variable
qui a un type "variable" parmi plusieurs.
\end{block}
La syntaxe est tr�s proche de celle utilis�e pour
les structures :
\begin{codeblock}{}
\vspace{-.3cm}
\lstset{escapeinside={��}}
\lstset{basicstyle=\scriptsize}
\begin{codeC}
union MonUnion {
 int entier;
 double reel;
 char chaine[100];
}
\end{codeC}
\vspace{-.3cm}
\end{codeblock}

Utilisation (comme pour les structures) :
\begin{codeblock}{}
\vspace{-.3cm}
\lstset{escapeinside={��}}
\lstset{basicstyle=\scriptsize}
\begin{codeC}
union MonUnion Var ;
Var.entier = 2 ;
\end{codeC}
\vspace{-.3cm}
\end{codeblock}

\begin{alertblock}{}
La taille de la variable est �gale � la taille du 
type le plus grand (dans l'exemple c'est \Verb|chaine| = 100 octets).
\end{alertblock}

\end{frame}

\begin{frame}[fragile]
\frametitle{Les unions}
Tous les champs de la variable de type \bvrb|union| partage
le m�me espace m�moire
\begin{figure}
\begin{tikzpicture} [
auto,
remember picture,
 block/.style    = { rectangle, draw=blue, black, 
                         text width=2.8cm, text centered,
                         font = \footnotesize,
                        rounded corners, minimum height=2em },
node distance=.5em,
]
\draw (0,0) node (var) [anchor = west] {Var};
\draw (0,-0.6) node (double) 
[anchor = west, rectangle, minimum height = 1.5em, minimum width = 4cm, inner sep = 0pt, fill = red, red]{};

\draw [very thick, red] 
($(double.west) + (0,-1.2)$) --
node [midway,below] {\Verb|double|}
 ($(double.east) + (0,-1.2)$) ;

\draw (0,-0.6) node (int) 
[anchor = west,rectangle, minimum height = 1.5em, minimum width = 2cm, inner sep = 0pt, fill = cyan, cyan]{};

\draw [very thick, cyan] 
($(int.west) + (0,-0.6)$) -- 
node [midway,below] {\Verb|int|}
($(int.east) + (0,-0.6)$) ;


\draw (0,-0.6) node (char) 
[anchor = west,rectangle, minimum height = 1.5em, minimum width = 10cm, inner sep = 0pt, draw=black]{};
\draw [very thick, black] 
($(char.west) + (0,-1.8)$) --
node [midway,below] {\Verb|char[100]|}
 ($(char.east) + (0,-1.8)$) ;


\end{tikzpicture}
\end{figure}

\begin{columns}
\column{0.6\textwidth}
\begin{codeblock}{}
\vspace{-.3cm}
\lstset{escapeinside={��}}
\lstset{basicstyle=\scriptsize}
\begin{codeC}
union MonUnion Var;
Var.entier = 200;
printf("Val int = %d\n",Var.entier);
Var.reel = 1200.05;
printf("Val double = %lf\n",Var.reel);
printf("Val int = %d\n",Var.entier);
\end{codeC}
\vspace{-.3cm}
\end{codeblock}

\column{0.33\textwidth}
\begin{termblock}{Test d'ex�cution}
%\vspace{-.3cm}
\lstset{escapeinside={��}}
\lstset{basicstyle=\scriptsize}
\begin{Verbatim}
Val int = 200
Val double = 1200.05
Val int = 858993459
\end{Verbatim}
\vspace{-.3cm}
\end{termblock}
\end{columns}


\end{frame}

\begin{frame}[fragile]
\frametitle{Adrese de la variable}
\begin{codeblock}{}
\vspace{-.3cm}
\lstset{escapeinside={��}}
\lstset{basicstyle=\scriptsize}
\begin{codeC}
union MonUnion variable;
printf("Adresse de l'union = %p\n",&variable);
printf("Adresse de la partie enti�re = %p\n",&variable.entier);
printf("Adresse de la partie r�elle = %p\n",&variable.reel);
\end{codeC}
\vspace{-.3cm}
\end{codeblock}

\begin{termblock}{Test d'ex�cution}
%\vspace{-.3cm}
\lstset{escapeinside={��}}
\lstset{basicstyle=\scriptsize}
\begin{Verbatim}
Adresse de l'union = 0023FF70
Adresse de la partie enti�re = 0023FF70
Adresse de la partie r�elle = 0023FF70
\end{Verbatim}
\vspace{-.3cm}
\end{termblock}

\end{frame}

\begin{frame}[fragile]



\begin{codeblock}{Mod�le de structure}
\vspace{-.3cm}
\lstset{escapeinside={��}}
\lstset{basicstyle=\scriptsize}
\begin{codeC}
enum type_t {ENTIER,CARACT};
struct touche {
  enum type_t type;
  union {
   int ent;
   char car;
  }
};
\end{codeC}
\vspace{-.3cm}
\end{codeblock}

\begin{codeblock}{Dans le programme :}
\vspace{-.3cm}
\lstset{escapeinside={��}}
\lstset{basicstyle=\scriptsize}
\begin{codeC}
struct touche t;
int n,res;
res=scanf("%d",&n);
if (res==0) {
  scanf("%c",&t.car);
  t.type=CARACT;
} 
else {
  t.ent=n;
  t.type=ENTIER;
}
\end{codeC}
\vspace{-.3cm}
\end{codeblock}


\end{frame}