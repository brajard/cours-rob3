% !TEX encoding = IsoLatin9
\section{Débogage}
\begin{frame}
  \begin{columns}
    \column{4.8cm}
    \tableofcontents[currentsection,hideothersubsections,currentsubsection]
    \column{7cm}
    \centering{
      \includegraphics[width=4cm]{fig/dijkstra.jpg}
      }

      \textit{"If debugging is the process of removing bugs, then
programming must be the process of putting them in."}\\
      \small{
        \hfill Edsger W. Dijkstra (1930-2002)}
    
  \end{columns}
  \end{frame}

\begin{frame}
\frametitle{Qu'est-ce qu'un bug ? (ou bogue)}
\begin{block}{Définition}
Un bug (ou en français un bogue) est un défaut dans le 
programme informatique du à une erreur de syntaxe ou à
\red{une erreur de conception du programme}
\end{block}

Quelques bugs informatiques célèbres :
\begin{itemize}
\item [https://youtu.be/gp_D8r-2hwk]{Ŀ'explosion en vol d'Ariane 5 lors du premier tir}
\item Le bug de l'an 2000
\item 99\% de vos futurs programmes.
\end{itemize}

\begin{alertblock}{}
Il est donc indispensable de déboguer ses programmes.
\end{alertblock}

\end{frame}

