\documentclass{beamer}

\usepackage[frenchb]{babel}
\usepackage[T1]{fontenc}
\usepackage[latin1]{inputenc}
\usepackage{hyperref}
\usepackage{listings}
\usepackage{fancyvrb}
\usepackage{tikz}
\usepackage{framed}
\usetheme{Boadilla}
\usecolortheme{dolphin}

\lstnewenvironment{codeC}
{ \lstset{language=C,
    otherkeywords={printf,scan}}
}
{}

\newenvironment<>{codeblock}[1]{%begin
  \setbeamercolor{block title}{fg=darkgray,bg=yellow}%
  \begin{block}{#1}}
  % \begin{codeC}}
  %  {\end{codeC}
{  
\end{block}}

\newenvironment<>{termblock}[1]{
    \setbeamercolor{block title}{fg=white,bg=lightgray}%
    \begin{block}{#1}}
%     \begin{Verbatim}}
{%\end{Verbatim}
\end{block}
}
%\newcommand{\output}[1]{

%%% Param�tres du cours (� r�gler)
%Num�ro du cours
\newcommand{\nb}{1}

\title[Cours n�\nb]{Cours n�\nb - Algorithmes et premiers pas en C}
\author[]{julien.brajard@upmc.fr}
\institute[Polytech' UPMC]{Polytech' UPMC}
\date{28 Septembre 2015}
\begin{document}
%%%%%%%%%%%%%%%%%%%%% SLIDES DE TITRE
\begin{frame}
\titlepage
\centering{
\url{http://australe.upmc.fr} (onglet EPU-C5-IGE Info Gen)}
\end{frame}
%%%%%%%%%%%%%%%%%%%%%
\begin{frame}
\frametitle{Plan du cours n�\nb}
\tableofcontents[hideallsubsections]
\end{frame}

%%%%%%%%%%%%%%%%%%%%% SECTION 1
\section{Les algorithmes}\label{section:1}
\begin{frame}
\begin{columns}
        \column{4.8cm}
            \tableofcontents[currentsection]
        \column{7cm}
        \centering{
            \includegraphics[width=7cm]{fig/Algorithm-sheldon.png}
            
                 \textit{ I believe I've isolateblblblblblblsblbslbslbsl
            sblbslblsblsblblsblbs
            lbslblbslsb d the algorithm for making friends.}
     
            
            \small{
            \hfill Sheldon Cooper, 
            
            \hfill in \textit{The Big Band Theory}, Season 2, Episode 13
            }
}

    \end{columns}

\end{frame}


%%%%%%%%%%%%%%%%%%%%%
\subsection{Introduction}
    \begin{frame}
    \frametitle{Pourquoi faire appel � des algorithmes ?}
    Pour automatiser des t�ches
    
    Exemples :
    \begin{itemize}
    \item M�tier � tisser\\
    \item M�thode de calcul � la main d'une division\\
    \item Recette de cuisine\\
    \item ...\\
    \end{itemize}
    \end{frame}
 
 %%%%%%%%%%%%%%%%%
 
    \begin{frame}
    \frametitle{Qu'est-ce qu'un algorithme ?}
    \begin{block}{D�finition}
    Un algorithme est un ensemble 
    ordonn� d'instructions simples
permettant de r�soudre un probl�me.
    \end{block}
    \end{frame}
    
 %%%%%%%%%%%%%%%%%%
 \subsection{Construction d'un algorithme}
%%%%%%%%%%%%%%%%%%%    
\section{La machine de Turing}
%%%%%%%%%%%%%%%%%%%%
 
  
\begin{frame}[fragile]
\frametitle{Un peu d'histoire...}
\begin{codeblock}{Test}
\begin{codeC}
for (int i = 0 ; i < n ; i ++) {
    //a comment
    printf("%d",i);
    }
\end{codeC}
\end{codeblock}

\begin{termblock}{test 2}
\lstset{escapeinside={��}}
\begin{lstlisting}
�\textbf{>>}�./a.out
�\color{darkgray}{\texttt{  Hello World}}�
\end{lstlisting}
\end{termblock}

 \begin{block}{Bloc standard}
blablabla
\end{block}
\end{frame}

\end{document}
