\ifdefined\ishandout
\documentclass[handout]{beamer}
\else
\documentclass{beamer}
\fi

\usepackage[frenchb]{babel}
\usepackage[T1]{fontenc}
\usepackage[latin1]{inputenc}
\usepackage{hyperref}
\usepackage{listings}
\usepackage{fancyvrb}
\usepackage{tikz}
\usepackage{framed}
\usepackage{algorithm}
\usepackage{algorithmic}
\usepackage{xcolor}
\usepackage{color, colortbl}
\usepackage{handoutWithNotes}
\usetikzlibrary{shapes.geometric}
\usetikzlibrary{shapes.arrows, chains}
\usetikzlibrary{arrows}
\usepackage{array}
\usetheme{Boadilla}

\ifdefined\ishandout
\pgfpagesuselayout{3 on 1 with notes}[a4paper,border shrink=5mm]
\usecolortheme{dove}
\else
\usecolortheme{dolphin}
\fi


\lstnewenvironment{codeC}
{ \lstset{language=C,
    otherkeywords={printf,scan}}
}
{}
\newcommand{\red}{\textcolor{red}}
%\newcommand \emph
%Default size : 12.8 cm * 9.6 cm

\newenvironment<>{codeblock}[1]{%begin
  \setbeamercolor{block title}{fg=darkgray,bg=yellow}%
  \begin{block}{#1}}
  % \begin{codeC}}
  %  {\end{codeC}
{  
\end{block}}

\newenvironment<>{termblock}[1]{
    \setbeamercolor{block title}{fg=white,bg=lightgray}%
    \begin{block}{#1}}
%     \begin{Verbatim}}
{%\end{Verbatim}
\end{block}
}
%\newcommand{\output}[1]{
\setbeamertemplate{navigation symbols}{}


%%% Param�tres du cours (� r�gler)
%Num�ro du cours
\newcommand{\nb}{1}

\title[Cours n�\nb]{Cours n�\nb - Algorithmes et premiers pas en C}
\author[]{julien.brajard@upmc.fr}
\institute[Polytech' UPMC]{Polytech' UPMC}
\date{28 Septembre 2015}
\begin{document}
%%%%%%%%%%%%%%%%%%%%% SLIDES DE TITRE
\begin{frame}
\titlepage
\centering{
\url{http://australe.upmc.fr} (onglet EPU-C5-IGE Info Gen)}
\end{frame}
%%%%%%%%%%%%%%%%%%%%%
\begin{frame}
\frametitle{Plan du cours n�\nb}
\tableofcontents[hideallsubsections]
\end{frame}

%%%%



%%%%%% SECTION 12
% !TEX encoding = IsoLatin9

%%%%%%%%%%%%%%%%%%%%% SECTION 1
\section{Les algorithmes}\label{section:1}
\begin{frame}
  \begin{columns}
    \column{4.8cm}
    \tableofcontents[currentsection,hideothersubsections]
    \column{7cm}
    \centering{
      \includegraphics[width=7cm]{fig/Algorithm-sheldon.png}
      
      \textit{``I believe I've isolated the algorithm for making friends.''}\\
      \small{
        \hfill Sheldon Cooper, 
        
        \hfill in \textit{The Big Band Theory}, Season 2, Episode 13
      }
    }
  \end{columns}
  
\end{frame}

%%%%%%%%%%%%%%%%%%%%% 

\subsection{Introduction}
\begin{frame}
  \frametitle{Pourquoi faire appel � des algorithmes ?}
  Pour automatiser des t�ches
  
  Exemples :
  \begin{itemize}
  \item M�tier � tisser\\
  \item M�thode de calcul � la main d'une division\\
  \item Recette de cuisine\\
  \item ...\\
  \end{itemize}
\end{frame}

%%%%%%%%%%%%%%%%% 

\begin{frame}
  \frametitle{Qu'est-ce qu'un algorithme ?}
  \begin{alertblock}{D�finition}
    Un algorithme est un ensemble 
    ordonn� d'instructions simples
    permettant de r�soudre un probl�me.
  \end{alertblock}
\end{frame}

%%%%%%%%%%%%%%%%%% 

\begin{frame}
\frametitle{Remarques}
  \begin{block}{Un algorithme n�cessite :}
    \begin{itemize}
    \item Des objets sur lesquels travailler,\\
    \item Un langage non ambigu,\\
    \item Des sp�cifications (description de l'algorithme).\\
    \end{itemize}
  \end{block}
  
  \begin{block}{}
    Il n'existe g�n�ralement pas un unique algorithme pour traiter un probl�me.
  \end{block}
\end{frame}

%%%%%%%%%%%%%%%%%

\begin{frame}
  \frametitle{Historique}
  \begin{description}
\item[3�me si�cle avant JC] \textit{Livre VII des El�ments d'Euclide}
 \begin{columns}[T]
    \column{8cm}
D�termination du plus grand diviseur commun entre deux nombres :
PGCD(12,8)=4\\

 \column{4cm}
    \centering{
      \includegraphics[width=4cm]{fig/Euclide.jpg}
}
\end{columns}
\item[8�me si�cle apr�s JC] \textbf{\textit{Al-Khawarizmi}} : M�thodes de r�solution d'�quations.
Son nom est � l'origine du mot "algorithme".
\end{description}
\end{frame}

%%%%%%%%%%%%%%%%%%%%%%%%%%%%%%%%%%%%%%%%%%%%%%%%%%%%%%%%%%%%%%%%%%%

\subsection{Construction d'un algorithme}

\begin{frame}
  \begin{columns}
    \column{4.8cm}
    \tableofcontents[currentsection,hideothersubsections,currentsubsection]
    \column{7cm}
    \centering{
      \includegraphics[width=4cm]{fig/construction.png}
     % \textit{``I believe I've isolated the algorithm for making friends.''}
     % \small{
     %   \hfill Sheldon Cooper, 
     %   \hfill in \textit{The Big Band Theory}, Season 2, Episode 13
%     }
    }
  \end{columns}
\end{frame}
 
%%%%%%%%%%%%%%%%%%%%%%%%%%%%%%%%%%%%%%%%%%%%%%%%%%%%%%%%%%%%%%%%%%%

\begin{frame}
  \frametitle{Construction d'un algorithme}
\begin{block}{Pour chaque probl�me, il vous est demand� de d�finir clairement :}
\begin{itemize}
\item Les �ventuelles \red{donn�es d'entr�e du probl�me} en pr�cisant leurs types et leur r�le,\\
\item les �ventuelles \red{donn�es de sortie du probl�me} en pr�cisant leurs types, \\
\item les diff�rentes \red{instructions} permettant d'obtenir les donn�es de sorties � partir
des donn�es d'entr�e.\\
\end{itemize}
\end{block}
\end{frame}

%%%%%%%%%%%%%%%%%%%%%%%%%%%%%%%%%%%%%%

\begin{frame}[fragile]
\frametitle{Exemple}
Algorithme qui d�termine le prix d'entr�e dans un mus�e (les mineurs payent moiti� prix)
\pause
\begin{columns}[t]
\column{6 cm}
\begin{exampleblock}{Donn�es}
\begin{algorithmic}[1]
\REQUIRE{age (entier)}\\
\COMMENT{age du client}
\ENSURE{tarif (d�cimal)}\\
\COMMENT{prix de l'entr�e}
\end{algorithmic}
\end{exampleblock}
\column{5 cm}
\pause

\begin{exampleblock}{Instructions}
\begin{algorithmic}[0]
\IF{age < 18}
\STATE tarif $\leftarrow$ 4
\ELSE
\STATE tarif $\leftarrow$ 8
\ENDIF
\RETURN{tarif}
\end{algorithmic}
\end{exampleblock}

\end{columns}
\end{frame}

%%%%%%%%%%%%%

\subsection{Structures de base d'un algorithme}

\begin{frame}
  \begin{columns}
    \column{4.8cm}
    \tableofcontents[currentsection,hideothersubsections,currentsubsection]
    \column{7cm}
    \centering{
      \includegraphics[width=7cm]{fig/lego.jpg}
     % \textit{``I believe I've isolated the algorithm for making friends.''}
     % \small{
     %   \hfill Sheldon Cooper, 
     %   \hfill in \textit{The Big Band Theory}, Season 2, Episode 13
%     }
    }
  \end{columns}
\end{frame}
 
%%%%%%%%%%%%%%%%%%%%%%%%%%%%%%%%%%%%%%%%%%%%%%%%%%%%%

\begin{frame}
\frametitle{Instructions �l�mentaires}
\begin{itemize}
\item \textbf{Op�rations arithm�tiques de base ($+$, $-$, $\times$, $/$, mod, ...)}\\
\item \textbf{Affectation de valeurs (ex : $x \leftarrow 2$)}\\
\item \textbf{Afficher, lire}\\
\item Structures conditionnelles\\
\item Structures r�p�titives\\
\end{itemize}


\end{frame}

%%%%%%%%%%%%%%%%%%%%%%%%%%%%%%%%%%%%%%%%%%%%%
\begin{frame}[fragile]
\frametitle{Structures conditionnelles 1/2}
\begin{columns}[t]
\column{6cm}
%\begin{block}

\begin{figure}
\begin{tikzpicture} [
    auto,
    decision/.style = { diamond, draw=blue, thick, fill=blue!20,
                        text width=1.5cm, text badly centered,
                        rounded corners, aspect=2 },
    block/.style    = { rectangle, draw=blue, thick, 
                        fill=blue!20, text width=1.5cm, text centered,
                        rounded corners, minimum height=2em },
    line/.style     = { draw, thick, ->, shorten >=2pt },
    node distance=1.5cm,
  ]
  \node (rac) {} ;
  \node (condition) [decision, below of=rac] {\red{condition}} ; 
  \node (center) [draw,circle,  thick, inner sep=2pt,minimum size= 0pt, radius = 2pt,below of =condition]{};
  \node (false) [block, left of=center,xshift=-3mm] {Action F};
  \node (true) [block, right of=center,xshift=3mm] {Action V};
   \node (end) [below of=center]{};
  % Define nodes in a matrix
  %\matrix [column sep=5mm, row sep=10mm] {
  %& \node (rac) {}; & \\
  %& \node [decision] (condition) {condition}; & \\
  %\node [block] (true) {Action V}; & & \node [block] (true) {Action V} ; \\
 % & \node (end) {}; \\
  %};
  
  % connect nodes
  \begin{scope}[every path/.style=line]
  \path (rac) -- (condition);
  \path (condition) -| node [near start, above] {vraie} (true);
  \path (condition) -| node [near start, above] {fausse} (false);
  \path (center) -- (end) ;
  %\path (false) -| (end);
  \end{scope}
  \draw[-, thick] (true) -- (center) ;
    \draw[-, thick] (false) -- (center) ;

\end{tikzpicture}
\end{figure}
%\end{block}
\column{5cm}

\begin{block}{}
\begin{algorithmic}[0]
\IF{\red{condition}}
\STATE Action V
\ELSE
\STATE Action F
\ENDIF
\end{algorithmic}
\end{block}

\begin{exampleblock}{Un exemple}
\begin{algorithmic}[0]
\IF{\red{age < 18}}
\STATE tarif $\leftarrow$ 4
\ELSE
\STATE tarif $\leftarrow$ 8
\ENDIF
\end{algorithmic}
\end{exampleblock}

\end{columns}
\end{frame}


%%%%%%%%%%%%%%%%%%%%%%%%%%%%%%%%%%%%%%%%%%%%%
\begin{frame}[fragile]
\frametitle{Structures conditionnelles 2/2}
\begin{columns}[t]
\column{6cm}
%\begin{block}

\begin{figure}
\begin{tikzpicture} [
    auto,
    decision/.style = { diamond, draw=blue, thick, fill=blue!20,
                        text width=1.5cm, text badly centered,
                        rounded corners, aspect=2 },
    block/.style    = { rectangle, draw=blue, thick, 
                        fill=blue!20, text width=1.5cm, text centered,
                        rounded corners, minimum height=2em },
    line/.style     = { draw, thick, ->, shorten >=2pt },
    node distance=1.5cm,
  ]
  \node (rac) {} ;
  \node (condition) [decision, below of=rac] {\red{condition}} ; 
  \node (center) [draw,circle,  thick, inner sep=2pt,minimum size= 0pt, radius = 2pt,below of =condition]{};
  %\node (false) [block, left of=center,xshift=-3mm] {Action F};
  \node (true) [block, right of=center,xshift=3mm] {Action V};
   \node (end) [below of=center]{};
  % Define nodes in a matrix
  %\matrix [column sep=5mm, row sep=10mm] {
  %& \node (rac) {}; & \\
  %& \node [decision] (condition) {condition}; & \\
  %\node [block] (true) {Action V}; & & \node [block] (true) {Action V} ; \\
 % & \node (end) {}; \\
  %};
  
  % connect nodes
  \begin{scope}[every path/.style=line]
  \path (rac) -- (condition);
  \path (condition) -| node [near start, above] {vraie} (true);
 % \path (condition) -| node [near start, above] {fausse} (false);
  \path (center) -- (end) ;
  %\path (false) -| (end);
  \end{scope}
  \draw[-, thick] (true) -- (center) ;
  \draw[-, thick] (condition) -- node [near start, left] {fausse}  (center) ;

\end{tikzpicture}
\end{figure}
%\end{block}
\column{5cm}

\begin{block}{}
\begin{algorithmic}[0]
\IF{\red{condition}}
\STATE Action V
\ENDIF
\end{algorithmic}
\end{block}

\begin{exampleblock}{Un exemple}
\begin{algorithmic}[0]
\IF{\red{age < 18}}
\PRINT "R�duction de 50\%"
\ENDIF
\end{algorithmic}
\end{exampleblock}

\end{columns}
\end{frame}


%%%%%%%%%%%%%%%%%%%%%%%%%%%%%%%%%%%%%%%%%%%%%
\begin{frame}[fragile]
\frametitle{Structures � choix multiple}
\begin{columns}[t]
\column{6cm}
%\begin{block}

\begin{figure}
\begin{tikzpicture} [
    auto,
    decision/.style = { diamond, draw=blue, thick, fill=blue!20,
                        text width=1.5cm, text badly centered,
                        rounded corners, aspect=2.2, scale=0.7 },
    block/.style    = { rectangle, draw=blue, thick, 
                        fill=blue!20, text width=2cm, text centered,
                        rounded corners, minimum height=2em , scale=0.8},
    line/.style     = { draw, thick, ->, shorten >=2pt },
    node distance=1.8cm,
  ]
  \node (rac) {} ;
  \node (selon) [block, below of=rac, yshift=7mm] {Selon \red{le cas}} ;
  \node (cas1) [decision, below of=selon] {Cas 1} ;
  \node (cas2) [decision, below of=cas1]  {Cas 2} ;
  \node (casn) [decision, below of=cas2, yshift=-3mm] {Cas n} ;
  \node (def)  [decision, below of=casn] {d�faut} ;
  \node (ac1) [block, right of = cas1, xshift= 11mm] {Action 1} ;
  \node (ac2) [block, right of = cas2, xshift= 11mm] {Action 2} ;
  \node (acn) [block, right of = casn, xshift= 11mm] {Action n} ;
  \node (acd) [block, right of = def, xshift =11mm] {Action par d�faut} ;
  \node (ce1) [draw, circle, thick, inner sep=2pt, minimum size =0pt, radius = 2pt, right of = ac1, xshift=-3mm]{};
  \node (ce2) [draw, circle, thick, inner sep=2pt, minimum size =0pt, radius = 2pt, right of = ac2, xshift=-3mm]{};
  \node (cen) [draw, circle, thick, inner sep=2pt, minimum size =0pt, radius = 2pt, right of = acn, xshift=-3mm]{};
  \node (ced) [draw, circle, thick, inner sep=2pt, minimum size =0pt, radius = 2pt, right of = acd, xshift=-3mm]{};
  \node (center)  [draw, circle, thick, inner sep=2pt, minimum size =0pt, radius = 2pt, below of = def, yshift=11mm]{};
  \node (out) [below of= center, yshift = 12mm]{};
%  \node (center) [draw,circle,  thick, inner sep=2pt,minimum size= 0pt, radius = 2pt,below of =condition]{};
 

  % connect nodes
  \begin{scope}[every path/.style=line]
  \path (rac) -- (selon);
  \path (selon) -- (cas1);
  \path (cas1) -- (cas2);
  \path (casn) -- (def);
  \path (cas1) -- (ac1) ;
  \path (cas2) -- (ac2) ;
  \path (casn) -- (acn) ;
  \path (def) -- (acd) ;
  \path (ced) |- (center);
  \path (center) -- (out) ;

  \end{scope}
  \draw[-, thick] (ac1) -- (ce1) ;
  \draw[-, thick] (ac2) -- (ce2) ;
  \draw[-, thick] (acn) -- (cen) ;
  \draw[-, thick] (acd) -- (ced) ;
  \draw[-, thick] (ce1) -- (ce2) ;
  \draw[-, thick] (ce2) -- (cen) ;
  \draw[-, thick] (cen) -- (ced) ;
  
  \draw[-, thick] (def) -- (center) ;

  
\draw[dotted, -, thick] (cas2) -- (casn) ;
\end{tikzpicture}
\end{figure}
%\end{block}
\column{5.1cm}

\begin{block}{}
\begin{algorithmic}[0]
\STATE{\textbf{Selon} \red{le cas}}
\STATE \hspace{0.8em} Cas 1 : Action 1 
\STATE \hspace{0.8em} Cas 2 : Action 2 
\STATE \hspace{0.8em} ... 
\STATE \hspace{0.8em} Cas n : Action n
\STATE \hspace{0.8em} D�faut : Action par d�faut
\STATE{\textbf{Fin Selon}}
\end{algorithmic}
\end{block}

\begin{exampleblock}{Un exemple}
\begin{algorithmic}[0]
\STATE{\textbf{Selon} \red{valeur de age}}
\STATE \hspace{0.8em} 0..6 : tarif $\leftarrow$ 0
\STATE \hspace{0.8em} 7..18 : tarif $\leftarrow$ 4
\STATE \hspace{0.8em} D�faut : tarif $\leftarrow$ 8
\STATE{\textbf{Fin Selon}}
\end{algorithmic}
\end{exampleblock}

\end{columns}
\end{frame}

%%%%%%%%%%%%%%%%%%%%%%%%%%%%%%%%%%%%%%%%%%%%%
\begin{frame}[fragile]
\frametitle{Structures r�p�titives 1/2}
\begin{columns}[t]
\column{6cm}
%\begin{block}

\begin{figure}
\begin{tikzpicture} [
    auto,
    decision/.style = { diamond, draw=blue, thick, fill=blue!20,
                        text width=3cm, inner sep = 1pt,text badly centered,
                        rounded corners, aspect=2, scale=0.9},
    block/.style    = { rectangle, draw=blue, thick, 
                        fill=blue!20, text width=1.5cm, text centered,
                        rounded corners, minimum height=2em },
    line/.style     = { draw, thick, ->, shorten >=2pt },
    node distance=2cm,
  ]
  \node (rac) {} ;
  %\node (bet) [draw,circle,  thick, inner sep=2pt,minimum size= 0pt, radius = 2pt,yshift=10mm, below of =rac]{};
  \node (condition) [decision, below of=rac, yshift=-3mm] {R�p�ter tant que \red{condition}} ; 
  \node (action) [block, below of = condition, yshift = -3mm] {Actions} ;
  \node (right) [right of = condition, xshift = 6mm] {} ;
   \node (left) [left of = condition, xshift = -7mm] {} ;

  \node (center) [below of =action, yshift=12mm]{};
  \node (end) [below of=center,yshift = 10mm]{};
  
  
  % connect nodes
  \begin{scope}[every path/.style=line]
%  \path (rac) -- (bet);
  \path (rac) -- (condition);
  \path (condition) -- node [near start, right] {vraie} (action);
 % \path (right.center) |- (center);
  \path (center.center) -- (end) ;
  \path (left.center) -- (condition) ;
  %\path (false) -| (end);
  \end{scope}
 % \draw[-, thick] (true) -- (center) ;
   % \draw[-, thick] (false) -- (center) ;
\draw[-, thick] (condition) -- node [near start, above] {fausse} (right.center);
\draw[-, thick] (action) -| (left.center);
\draw [-, thick] (right.center) |- (center.center);

\end{tikzpicture}
\end{figure}
%\end{block}
\column{5cm}

\begin{block}{}
\begin{algorithmic}[0]
\WHILE{\red{condition}}
\STATE Actions
\ENDWHILE
\end{algorithmic}
\end{block}

\begin{alertblock}{}
Les actions peuvent ne \textbf{pas} �tre ex�cut�es
si la condition est fausse au d�part.
\end{alertblock}
\begin{exampleblock}{Un exemple}
\begin{algorithmic}[0]
\WHILE{\red{tickets>0}}
\STATE vendre un ticket
\ENDWHILE
\end{algorithmic}
\end{exampleblock}

\end{columns}
\end{frame}


%%%%%%%%%%%%%%%%%%%%%%%%%%%%%%%%%%%%%%%%%%%%%


%%%%%%%%%%%%%%%%%%%%%%%%%%%%%%%%%%%%%%%%%%%%%
\begin{frame}[fragile]
\frametitle{Structures r�p�titives 2/2}
\begin{columns}[t]
\column{6cm}
%\begin{block}

\begin{figure}
\begin{tikzpicture} [
    auto,
    decision/.style = { diamond, draw=blue, thick, fill=blue!20,
                        text width=3cm, inner sep = 1pt,text badly centered,
                        rounded corners, aspect=2, scale=0.9},
    block/.style    = { rectangle, draw=blue, thick, 
                        fill=blue!20, text width=1.5cm, text centered,
                        rounded corners, minimum height=2em },
    line/.style     = { draw, thick, ->, shorten >=2pt },
    node distance=2cm,
  ]
  \node (rac) {} ;
%  \node (bet) [draw,circle,  thick, inner sep=2pt,minimum size= 0pt, radius = 2pt,yshift=10mm, below of =rac]{};
 \node (action) [block, below of = rac, yshift = 10mm] {Actions} ;
  \node (condition) [decision, below of=action, yshift=-3mm] {R�p�ter tant que \red{condition}} ; 
 
%  \node (right) [right of = condition, xshift = 6mm] {} ;
   \node (left) [left of = condition, xshift = -5mm] {} ;

  \node (center) [below of =condition, yshift=12mm]{};
  \node (end) [below of=center,yshift = 5mm]{};
  
  
  % connect nodes
  \begin{scope}[every path/.style=line]
%  \path (rac) -- (bet);
  \path (rac) -- (action);
  \path (left.center) |- node [near start, right] {vraie} (action);
  \path (action) -- (condition);
  \path (condition) -- node [near start, right] {fausse} (end) ;
  %\path (left.center) -- (bet) ;
  %\path (false) -| (end);
  \end{scope}
 % \draw[-, thick] (true) -- (center) ;
 \draw[-, thick] (condition) -- (left.center) ;
%\draw[-, thick] (condition) -- node [near start, above] {fausse} (right.center);
%\draw[-, thick] (action) -| (left.center);
%\draw [-, thick] (right.center) |- (center.center);

\end{tikzpicture}
\end{figure}
%\end{block}
\column{5cm}

\begin{block}{}
\begin{algorithmic}[0]
\REPEAT
\STATE Actions
\UNTIL{\red{condition}}
\end{algorithmic}
\end{block}

\begin{alertblock}{}
Les actions sont ex�cut�es au moins une fois.
\end{alertblock}
\begin{exampleblock}{Un exemple}
\begin{algorithmic}[0]
\REPEAT
\STATE vendre un ticket
\UNTIL{ticket>0}
\end{algorithmic}
\end{exampleblock}

\end{columns}
\end{frame}


%%%%%%%%%%%%%%%%%%%%%%%%%%%%%%%%%%%%%%%%%%%%%


%%%%%%%%%%%%%%%%%%%%%%%%%%%%%%%%%%%%%%%%%%%%%
\begin{frame}[fragile]
\frametitle{Structures it�ratives}
\begin{columns}[t]
\column{6cm}
%\begin{block}

\begin{figure}
\begin{tikzpicture} [
    auto,
    decision/.style = { diamond, draw=blue, thick, fill=blue!20,
                        text width=3cm, inner sep = 1pt,text badly centered,
                        rounded corners, aspect=2, scale=0.9},
    block/.style    = { rectangle, draw=blue, thick, 
                        fill=blue!20, text width=1.5cm, text centered,
                        rounded corners, minimum height=2em },
    line/.style     = { draw, thick, ->, shorten >=2pt },
    node distance=2cm,
  ]
  \node (rac) {} ;
\node(condition)[decision, below of = rac]{R�p�ter \red{n} fois};
\node(action)[block, below of = condition]{Actions};
%  \node (bet) [draw,circle,  thick, inner sep=2pt,minimum size= 0pt, radius = 2pt,yshift=10mm, below of =rac]{};
% \node (action) [block, below of = rac, yshift = 10mm] {Actions} ;
 % \node (condition) [decision, below of=action, yshift=-3mm] {R�p�ter tant que \red{condition}} ; 
 
  \node (right) [right of = condition, xshift = 6mm] {} ;
  \node (left) [left of = condition, xshift = -6mm] {} ;

  \node (center) [below of =action,yshift=12mm]{};
  \node (end) [below of=center, yshift=9mm]{};
  
  
  % connect nodes
  \begin{scope}[every path/.style=line]
    \path (rac) -- (condition);
    \path (left.center) -- (condition);
  %  \path (right.center) -- (condition);
    \path (center.center) -- (end);
% \path (left.center) |- node [near start, right] {vraie} (action);
    \path (condition) -- (action);
    % \path (condition) -- node [near start, right] {fausse} (end) ;
    % \path (left.center) -- (bet) ;
    % \path (false) -| (end);
    
  \end{scope}
 % \draw[-, thick] (true) -- (center) ;
 \draw[-, thick] (action) -| (left.center) ;
 \draw[-, thick] (condition) --  node [midway, above, font = \scriptsize] {compteur>n} (right.center) ;
 \draw[-, thick] (right.center) |-(center.center) ;

%\draw[-, thick] (condition) -- node [near start, above] {fausse} (right.center);
%\draw[-, thick] (action) -| (left.center);
%\draw [-, thick] (right.center) |- (center.center);

\end{tikzpicture}
\end{figure}
%\end{block}
\column{6cm}

\begin{block}{}
\begin{algorithmic}[0]
\FOR {i = 1 � n}
\STATE Actions
\ENDFOR
\end{algorithmic}
\end{block}

\begin{alertblock}{}
On conna�t le nombre d'it�rations � l'avance.
\end{alertblock}
\begin{exampleblock}{Un exemple}
\begin{algorithmic}[0]
\FOR {i = 1 � nbre\_guide}
\IF {place pour guide i>0}
\PRINT {"Il reste des places pour le guide i"}
\ENDIF
\ENDFOR
\end{algorithmic}
\end{exampleblock}

\end{columns}
\end{frame}


%%%%%%%%%%%%%%%%%%%%%%%%%%%%%%%%%%%%%%%%%%%%%


\subsection{Tester un algorithme}

\begin{frame}
  \begin{columns}
    \column{4.8cm}
    \tableofcontents[currentsection,hideothersubsections,currentsubsection]
    \column{7cm}
    \centering{
      \includegraphics[width=7cm]{fig/dominos.jpg}
     % \textit{``I believe I've isolated the algorithm for making friends.''}
     % \small{
     %   \hfill Sheldon Cooper, 
     %   \hfill in \textit{The Big Band Theory}, Season 2, Episode 13
%     }
    }
  \end{columns}
\end{frame}

%%%%%%%%%%%%%%%%%%%%%%%%%%%%%%%%%%%%%%%%%%%%%%%%%%%%%%%%%%%%%%%%%%

\begin{frame}
\frametitle{Jeu d'essais}
\begin{itemize}
\item Faire fonctionner l'algorithme avec des valeurs.
\item Traiter tous les cas possibles.
\item En particulier les cas pouvant n�cessaiter � traitement sp�cifique.
\end{itemize}
\end{frame}

%%%%%%%%%%%%%%%%%%%%%%%%%%%%%%%%%%%%%%%%%%%%%%%%%%%%%%%%%%%%%%%%%%

\begin{frame}[fragile]
\frametitle{Un exemple :}
\framesubtitle{Calcul de la valeur absolue d'un nombre}
\begin{exampleblock}{Donn�es}
\begin{algorithmic}[1]
\REQUIRE{X (entier)}\\
\COMMENT{nombre dont on veut calculer la valeur absolue}
\ENSURE{Val\_abs (entier)}\\
\COMMENT{valeur absolue de X}
\end{algorithmic}
\end{exampleblock}
\begin{columns}
\column{0.45\textwidth}

\begin{exampleblock}{Instructions}
\begin{algorithmic}[0]
\IF{$X \geq 0$}
\STATE Val\_abs $\leftarrow$ X
\ELSE
\STATE  Val\_abs $\leftarrow$ -X
\ENDIF
\RETURN{ Val\_abs}
\end{algorithmic}
\end{exampleblock}

\column{0.45\textwidth}
\pause
\begin{exampleblock}{Jeu d'essais}
\begin{itemize}
\item X $\leftarrow$ un entier n�gatif
\item X $\leftarrow$ un entier positif
\item X $\leftarrow$ 0
\end{itemize}
\end{exampleblock}
\end{columns}

\end{frame}

%%%%%%%%%%%%%%%%%%%%%%%%%%%%%%%%%%%%%%%%%%%%%


\subsection{Exemples}

\begin{frame}
  \begin{columns}
    \column{4.8cm}
    \tableofcontents[currentsection,hideothersubsections,currentsubsection]
    \column{7cm}
    \centering{
      %\includegraphics[width=7cm]{fig/dominos.jpg}
     % \textit{``I believe I've isolated the algorithm for making friends.''}
     % \small{
     %   \hfill Sheldon Cooper, 
     %   \hfill in \textit{The Big Band Theory}, Season 2, Episode 13
%     }
    }
  \end{columns}
\end{frame}

%%%%%%%%%%%%%%%%%%%%%%%%%%%%%%%%%%%%%%%%%%%%%%%%%%%%%%%%%%%%%%%%%%

\begin{frame}
\frametitle{Exemple 1}
Un colis est conforme si ses trois dimensions
($L \times l \times h$ avec $h \leq l \leq L$) v�rifient
les conditions suivantes : 
\begin{itemize}
\item Un rectangle de 10cm $\times$ 7cm doit pouvoir s'inscrire
dans une face du colis,\\
\item $L + l + h \leq 100$ cm,\\
\item $L \leq 60 cm.$\\
\end{itemize}

\begin{block}{Exercice}
Ecrire l'algorithme permettant de savoir si un colis est conforme ou non.
\end{block}
\end{frame}

\ifdefined\correction
\begin{frame}[fragile]
\frametitle{Exemple 1 -corrig�}
\begin{block}{}
\begin{algorithmic}[0]
\REQUIRE{~\\L (r�el),l (r�el),h (r�el)}
\COMMENT{longeur,largeur,hauteur}
\ENSURE{Pas de sortie}

\IF {($L>10$ \textbf{et} $l>7$)\\ 
\textbf{et} $L+l+h \leq 100$\\
\textbf{et} $L \leq 60$
}
\PRINT "Colis conforme"
\ENDIF
\end{algorithmic}
\end{block}
\end{frame}
\fi

\begin{frame}[fragile]
\frametitle{Exemple 2}
\framesubtitle{Que fait cet algo ?}
\begin{algorithmic}[0]
\REQUIRE{~\\X, tableau � 2 dimension de n$\times$n r�els,\\
v, vecteur de n r�els}
\ENSURE{s, vecteur de n r�els}
\FOR{i variant de 1 � n}
\STATE s(i) $\leftarrow$ 0
\FOR {j variant de 1 � n}
\STATE s(i) $\leftarrow$ s(i) + X(i,j).v(j)
\ENDFOR
\ENDFOR
\RETURN{s}
\end{algorithmic}
\end{frame}

\ifdefined\correction
\begin{frame}
\frametitle{Corrig�}
\begin{block}{Multiplication Matrice-vecteur}
\begin{equation*}
s = X.v
\end{equation*}

\begin{equation*}
s_i = \sum_{j=1}^n X_{ij}.v_j
\end{equation*}


\end{block}
\end{frame}
\fi


% !TEX encoding = IsoLatin9

\section{La machine de Turing}
\begin{frame}
  \begin{columns}
    \column{4.8cm}
    \tableofcontents[currentsection,hideothersubsections]
    \column{7cm}
    \centering{
      \includegraphics[width=4cm]{fig/turing.jpg}
      
      \textit{``A computer would deserve to be called intelligent
if it could deceive a human into believing that it was human.''}\\
      \small{
        \hfill Alan Turing (1912-1954) 
              }
    }
  \end{columns}
  
\end{frame}

\begin{frame}
\frametitle{Un peu d'histoire...}
\begin{description}
%[font=\color{red}]
\item[1900 : ]Probl�mes d'Hilbert (2�me probl�me : Est-ce que l'arithm�tique est coh�rente ?)\\
\item[1931 : ]G�del d�montre que l'arithm�tique est incompl�te. Cela est li� au probl�me de la d�cision.\\
\item[1936 : ]Introduction de la machine de Turing par Alan Turing � l'�ge de 24 ans pour montrer
que certais probl�mes ne peuvent pas �tre r�solus par un algorithme (c'est � dire par la machine de Turing).\\
\item[1942 : ]Participation au d�codage d'Enigma.\\
\item[1946 : ]Cr�ation de l'ENIAC (Electronic Numerical Integrator and Computer).\\
\item[1954 : ]Mort de Turing.\includegraphics[scale=0.03]{./fig/apple.jpg}\\
\end{description}
\end{frame}

\begin{frame}
\frametitle{D�finition de la machine de Turing}
\begin{columns}

\column{.4\textwidth}
\begin{enumerate}
\item<1-|alert@1> Un ruban infini contenant des symboles. \tikz[remember picture,baseline=-.5ex] \coordinate(it1);
\item<2-|alert@2> Un t�te de lecture/�criture. \tikz[remember picture,baseline=-.5ex] \coordinate(it2);
\item<3-|alert@3> Un registre d'�tat. \tikz[remember picture,baseline=-.5ex] \coordinate(it3);
\item<4-|alert@4> Une table d'actions. \tikz[remember picture,baseline=-.5ex] \coordinate(it4);
\item<5-|alert@5> Un �tat et une position de d�part sur le ruban.\\
\end{enumerate}
\column{.55\textwidth}
\begin{figure}
\centering
\begin{tikzpicture}[remember picture]
\tikzstyle{every path}=[very thick]

\edef\sizetape{0.7cm}
\tikzstyle{tmtape}=[draw,minimum size=\sizetape]
\tikzstyle{tmhead}=[arrow box,draw,minimum size=.5cm,arrow box
arrows={east:.25cm, west:0.25cm}]

%% Draw TM tape
\begin{scope}[start chain=1 going right,node distance=-0.15mm]
    \node [on chain=1,tmtape,draw=none,xshift=1mm] {\tiny{$\ldots$}};
    \node [on chain=1,tmtape] {0};
    \node [on chain=1,tmtape] {1};
    \node [on chain=1,tmtape] {1};
    \node [on chain=1,tmtape] {0};
    \node [on chain=1,tmtape] (input){0};
    \node [on chain=1,tmtape] {0};
    \node [on chain=1,tmtape] {0};
    \node [on chain=1,tmtape] {0};
    \node [on chain=1,tmtape,draw=none,xshift=-1mm] (right) {\tiny{$\ldots$}};
\end{scope}
\node [tmhead,yshift=-.3cm] at (input.south) (head) {$e_1$};
\end{tikzpicture}
\end{figure}


\end{columns}
\begin{table}
\footnotesize
\begin{tabular}{|c|c|c|c|c|}
\hline
\textbf{Ancien �tat} & \textbf{Symbole lu} & 
\textbf{Symbole\tikz[remember picture,baseline=-1ex] \coordinate(tab);   �crit} & \textbf{Mouvement} &   \textbf{Nouvel �tat} \\
\hline
& & & &\\
\end{tabular}
\end{table}

\begin{tikzpicture}[remember picture,overlay,scale=0.8]
\draw <1|handout:0> (it1) edge[bend left, ->, very thick,shorten >=2pt,color=red ] (input.north);
\draw <2-> (it1) edge[bend left, ->, very thick,shorten >=2pt ] (input.north);

\draw<2|handout:0> (it2) edge[bend right, ->, very thick,shorten >=2pt,color=red] (head.west) ;
\draw<3-> (it2) edge[bend right, ->, very thick,shorten >=2pt] (head.west) ;

\draw<3|handout:0> (it3) edge[bend right, ->, very thick,shorten >=5pt,color=red] (head.center) ;
\draw<4-> (it3) edge[bend right, ->, very thick,shorten >=5pt] (head.center) ;

\draw<4|handout:0> (it4)   edge[bend left, ->, very thick,shorten >=5pt,color=red](tab);
\draw<5-> (it4)   edge[bend left, ->, very thick,shorten >=5pt](tab);
\end{tikzpicture}

\end{frame}


\begin{frame}
\frametitle{Fonctionnement d'une machine de Turing}
\begin{columns}

\column{.4\textwidth}

\begin{enumerate}
\footnotesize
\item<1-|alert@1>La t�te de lecture lit un symbole $s$. \tikz[remember picture,baseline=-.5ex] \coordinate(it1);
\item<2-|alert@2>En fonction de $s$ et de son �tat, la t�te �crit un symbole $s'$ \tikz[remember picture,baseline=-.5ex] \coordinate(it2);
\item<3-|alert@3>En fonction de son �tat la t�te se d�place d'un cran.  \tikz[remember picture,baseline=-.5ex] \coordinate(it3);
\item<4-|alert@4>La machine change d'�tat. \tikz[remember picture,baseline=-.5ex] \coordinate(it4);
\end{enumerate}


\column{.55\textwidth}
\begin{figure}
\centering
\begin{tikzpicture}[remember picture]
\tikzstyle{every path}=[very thick]

\edef\sizetape{0.7cm}
\tikzstyle{tmtape}=[draw,minimum size=\sizetape]
\tikzstyle{tmhead}=[arrow box,draw,minimum size=.5cm,arrow box
arrows={east:.25cm, west:0.25cm}]

%% Draw TM tape
\begin{scope}[start chain=1 going right,node distance=-0.15mm]
    \node [on chain=1,tmtape,draw=none,xshift=1mm] {\tiny{$\ldots$}};
    \node [on chain=1,tmtape] {0};
    \node [on chain=1,tmtape] {0};
    \node [on chain=1,tmtape] {0};
    \node [on chain=1,tmtape] {0};
    \node [on chain=1,tmtape] (input){\alt<2-|handout:0>{\textcolor<2>{red}{1}}{\textcolor<1>{red}{0}}};
    \node [on chain=1,tmtape] (step) {0};
    \node [on chain=1,tmtape] {0};
    \node [on chain=1,tmtape] {0};
    \node [on chain=1,tmtape,draw=none,xshift=-1mm] (right) {\tiny{$\ldots$}};
\end{scope}
\visible<1-2> {\node [tmhead,yshift=-.3cm] at (input.south) (head) {$e_1$};}
\visible<3-|handout:0> {\node [tmhead,yshift=-.3cm] at (step.south) (head) {\alt<4->{\textcolor{red}{$e_2$}}{$e_1$}};}

\end{tikzpicture}
\end{figure}


\end{columns}
\begin{table}
\footnotesize
\begin{tabular}{|c|c|c|c|c|}
\hline
\textbf{Ancien �tat} & \textbf{Symbole lu} & 
\textbf{Symbole\tikz[remember picture,baseline=-1ex] \coordinate(tab);   �crit} & \textbf{Mouvement} &   \textbf{Nouvel �tat} \\
\hline
\textcolor<1-2>{red}{$e_1$} & \textcolor<1-2>{red}{0} & \textcolor<2>{red}{1} &  \textcolor<3>{red}{D} &  \textcolor<4>{red}{$e_2$}\\
\hline
$e_2$ & 0 & 1 & D & $e_3$\\
\hline
$e_3$ & 0 & 1 & D & $e_4$\\
\hline
$e_4$ & 0 & 1 & D & $e_5$\\
\hline
$e_5$ & 0 & 1 & G & $e_4$\\
\hline
$e_1$ & 1 & 0 & G & ARRET\\
\hline
$e_2$ & 1 & 1 & G & $e_1$\\
\hline
$e_3$ & 1 & 0 & G & $e_2$\\
\hline
$e_4$ & 1 & 0 & G & $e_3$\\
\hline
$e_5$ & 1 & 0 & D & $e_4$\\
\hline
\end{tabular}
\end{table}

\end{frame}

\end{document}

%%%%%%%%%%%%%%%%%%%%% SECTION 1
\section{Les algorithmes}\label{section:1}
\begin{frame}
\begin{columns}
        \column{4.8cm}
            \tableofcontents[currentsection]
        \column{7cm}
        \centering{
            \includegraphics[width=7cm]{fig/Algorithm-sheldon.png}
            
                 \textit{ I believe I've isolateblblblblblblsblbslbslbsl
            sblbslblsblsblblsblbs
            lbslblbslsb d the algorithm for making friends.}
     
            
            \small{
            \hfill Sheldon Cooper, 
            
            \hfill in \textit{The Big Band Theory}, Season 2, Episode 13
            }
}

    \end{columns}

\end{frame}


%%%%%%%%%%%%%%%%%%%%%
\subsection{Introduction}
    \begin{frame}
    \frametitle{Pourquoi faire appel � des algorithmes ?}
    Pour automatiser des t�ches
    
    Exemples :
    \begin{itemize}
    \item M�tier � tisser\\
    \item M�thode de calcul � la main d'une division\\
    \item Recette de cuisine\\
    \item ...\\
    \end{itemize}
    \end{frame}
 
 %%%%%%%%%%%%%%%%%
 
    \begin{frame}
    \frametitle{Qu'est-ce qu'un algorithme ?}
    \begin{block}{D�finition}
    Un algorithme est un ensemble 
    ordonn� d'instructions simples
permettant de r�soudre un probl�me.
    \end{block}
    \end{frame}
    
 %%%%%%%%%%%%%%%%%%
 \subsection{Construction d'un algorithme}
%%%%%%%%%%%%%%%%%%%    
\section{La machine de Turing}
%%%%%%%%%%%%%%%%%%%%
 
  
\begin{frame}[fragile]
\frametitle{Un peu d'histoire...}
\begin{codeblock}{Test}
\begin{codeC}
for (int i = 0 ; i < n ; i ++) {
    //a comment
    printf("%d",i);
    }
\end{codeC}
\end{codeblock}

\begin{termblock}{test 2}
\lstset{escapeinside={��}}
\begin{lstlisting}
�\textbf{>>}�./a.out
�\color{darkgray}{\texttt{  Hello World}}�
\end{lstlisting}
\end{termblock}

 \begin{block}{Bloc standard}
blablabla
\end{block}
\end{frame}


\begin{frame}[fragile]
\frametitle{essai}
\begin{columns}
\column{6cm}
\begin{block}

\begin{figure}
\begin{tikzpicture} [
    auto,
    decision/.style = { diamond, draw=blue, thick, fill=blue!20,
                        text width=5em, text badly centered,
                        inner sep=1pt, rounded corners },
    block/.style    = { rectangle, draw=blue, thick, 
                        fill=blue!20, text width=10em, text centered,
                        rounded corners, minimum height=2em },
    line/.style     = { draw, thick, ->, shorten >=2pt },
  ]
   \matrix [column sep=-10mm, row sep=10mm] {
                    & \node [text centered] (x) {$\mathbf{X}$};            & \\
                    & \node (null1) {};                                    & \\
                    & \node [block] (doa) {\textsf{DoAE}($\mathbf{X}$)};   & \\
  	               \node(null3){}; & \node [decision] (uiddes)
                        {\textsf{UID}($\hat{\mathbf{X}}$)};
                                  & \node[text centered](tra){$\mathbf{i}$}; \\
                  & \node [block] (track) {\textsf{DoAT}($\mathbf{x}$)}; & \\
                    & \node [block] (pesos)
                        {\textsf{BF}(DoA$_{\mathrm{T}}$,DoAs)};            & \\
                    & \node [block] (filtrado)
                        {\textsf{SF}($\mathbf{w}$,$\mathbf{x}$)};          & \\
                    & \node [text centered] (xf) {$\hat{x}(t)$ };          & \\
  };
  % connect all nodes defined above
 \begin{scope} [every path/.style=line]
    \path (x)        --    (doa);
    \path (doa)      --    node [near start] {DoAs} (uiddes);
    \path (tra)      --    (uiddes);
    \path (uiddes)   --++  (-3,0) node [near start] {no} |- (null1);
    \path (uiddes)   --    node [near start] {DoA} (track);
    \path (track)    --    node [near start] {DoA$_{\mathrm{T}}$} (pesos);
    \path (pesos)    --    node [near start] {\textbf{w}} (filtrado);
    \path (filtrado) --    (xf);
  
  \end{scope}
\end{tikzpicture}
\end{figure}
\end{block}
\column{3cm}
\begin{block}{bulbul}
\end{block}
\end{columns}
\end{frame}

\end{document}
