% !TEX encoding = IsoLatin9

%%%%%%%%%%%%%%%%%%%%% SECTION 1
\section{La compilation}
\begin{frame}
  \begin{columns}
    \column{4.8cm}
    \tableofcontents[currentsection,hideothersubsections]
    \column{7cm}
    
  \end{columns}
  
\end{frame}

%%%%%%%%%%%%%%%%%%%%%%%%%%%%%%%%%%%%%%%%%%%
%                  FRAME 1                %
%%%%%%%%%%%%%%%%%%%%%%%%%%%%%%%%%%%%%%%%%%%

\begin{frame}[t,fragile]
\frametitle{Les 4 �tapes de la compilation}

%%%%%%%%%%%%%% ALL SLIDES %%%%%%%%%%%%%%%%%%
\begin{figure}[t]
\centering
\begin{tikzpicture} [
  block/.style    = { rectangle, draw=blue, thick, 
                      fill=blue!20, text width=1.8cm, text centered,
                      rounded corners, minimum height=2em },
 ablock/.style    = { rectangle, draw=red, thick, 
                      fill=red!20, text width=1.8cm, text centered,
                      rounded corners, minimum height=2em },
 line/.style     = { draw, very thick, ->, shorten >=1pt },
 aline/.style     = { draw, color=red,very thick, ->, shorten >=1pt },
 label/.style    = {midway,anchor=north,yshift=-0.3cm, text width=2.4cm,text centered},
  node distance = 2.5cm,
]
\node <1-> (cfile) [block] {\Verb|Bonjour.c|};
\node <1> (ifile) [block, right of = cfile] {\Verb|Bonjour.i|};
\node <2-> (ifile) [ablock, right of = cfile] {\Verb|Bonjour.i|};
\node (sfile) [block, right of = ifile] {\Verb|Bonjour.s|};
\node (ofile) [block, right of = sfile] {\Verb|Bonjour.o|};
\node (efile) [block, right of = ofile] {\Verb|Bonjour|};

\path <1> [line] (cfile) --  node[label]{preprocessing}(ifile) ;
\path <2-> [aline] (cfile) --  node[label]{preprocessing}(ifile) ;
\path[line] (ifile) -- node[label]{compilation}(sfile) ;
\path[line] (sfile) -- node[label]{assemblage}(ofile) ;
\path[line] (ofile) -- node[label]{�dition des liens}(efile) ;

\end{tikzpicture}
\end{figure}
%%%%%%%%%%%%%% SLIDE 1 %%%%%%%%%%%%%%%%%%
\begin{onlyenv}<1>
\vspace{-0.9cm}
\begin{columns}[t]
\column{.33\textwidth}
\begin{codeblock}{\Verb|bonjour.c|\hfill\tikz[remember picture,baseline=-.5ex] \coordinate(b1);}
\vspace{-.3cm}
\tikz[remember picture,baseline=0ex] (tab) {X};
\lstset{escapeinside={��}}
\lstset{basicstyle=\scriptsize}
\begin{codeC}
#include <stdio.h>

int main () {
 // Affiche "bonjour"
 printf("bonjour\n");
}
\end{codeC}
\end{codeblock}

\column{.38\textwidth}
\begin{termblock}{\tikz[remember picture,baseline=-.5ex] \coordinate(b2);\Verb|Terminal|}
\vspace{-.3cm}
\lstset{escapeinside={��}}
\lstset{basicstyle=\scriptsize}
\begin{lstlisting}
�\textbf{>>}�gcc Bonjour.c -o Bonjour
\end{lstlisting}
\centering{\tikz[remember picture,baseline=-0.5ex] \coordinate(b2_south);}
\vspace{-.6cm}
\end{termblock}
\vspace{0.8cm}
\begin{block}{}
%\centering{\tikz[remember picture,baseline=-.5ex] \coordinate(b3);}\\
Fichier Ex�cutable \Verb|Bonjour|
\end{block}
\end{columns}

\begin{alertblock}{}
Par d�faut \Verb|gcc| effectue les 4 �tapes.
\end{alertblock}

\begin{tikzpicture}[remember picture,overlay]
\draw (b1) edge[->, very thick,shorten >=10pt, shorten <=10pt] (b2);
\draw ($(b2_south)+(0,0.6)$) edge[->, very thick,shorten >=10pt, shorten <=10pt] ++(0,-1.6);

\end{tikzpicture}

\end{onlyenv}


%%%%%%%%%%%%%% SLIDE 2 %%%%%%%%%%%%%%%%%%
\begin{onlyenv}<2>
\vspace{-0.9cm}
\begin{columns}[t]
\column{.33\textwidth}
\begin{codeblock}{\Verb|bonjour.c|\hfill\tikz[remember picture,baseline=-.5ex] \coordinate(b1_2);}
\vspace{-.3cm}
\lstset{escapeinside={��}}
\lstset{basicstyle=\scriptsize}
\begin{codeC}
#include <stdio.h>

int main () {
 // Affiche "bonjour"
 printf("bonjour\n");
}
\end{codeC}
\end{codeblock}

\column{.6\textwidth}
\begin{termblock}{\tikz[remember picture,baseline=-.5ex] \coordinate(b2_2);\Verb|Terminal|}
\vspace{-.3cm}
\lstset{escapeinside={��}}
\lstset{basicstyle=\scriptsize}
\begin{lstlisting}
�\textbf{>>}�gcc -S Bonjour.c -o Bonjour
\end{lstlisting}
\centering{\tikz[remember picture,baseline=-0.5ex] \coordinate(b2_2_south);}
\vspace{-.6cm}
\end{termblock}
\vspace{0.2cm}
\begin{codeblock}{\Verb|bonjour.i|}
\vspace{-.3cm}
\lstset{escapeinside={��}}
\lstset{basicstyle=\scriptsize}
\begin{codeC}
(...) extern int printf 
 (__const char *__restrict__format, _);
(...)
# 1 "bonjour.c"
int main () {
 printf("bonjour\n");
}
\end{codeC}
\end{codeblock}
\end{columns}

\begin{tikzpicture}[remember picture,overlay]
\draw (b1_2) edge[->, very thick,shorten >=3pt, shorten <=2pt] (b2_2);
\draw ($(b2_2_south)+(0,0.6)$) edge[->, very thick , shorten <=10pt] ++(0,-0.7);

\end{tikzpicture}

\end{onlyenv}

\end{frame}